\documentclass{beamer}
%
% Choose how your presentation looks.
%
% For more themes, color themes and font themes, see:
% http://deic.uab.es/~iblanes/beamer_gallery/index_by_theme.html
%
\mode<presentation>
{
  \usetheme{default}      % or try Darmstadt, Madrid, Warsaw, ...
  \usecolortheme{crane} % or try albatross, beaver, crane, ...
  \usefonttheme{default}  % or try serif, structurebold, ...
  \setbeamertemplate{navigation symbols}{}
  \setbeamertemplate{caption}[numbered]
} 

\usepackage[english]{babel}
\usepackage[utf8x]{inputenc}
\usepackage{hhline}
\usepackage[font=small,labelfont=bf]{caption}

\title[Your Short Title]{Forecasting Energy Peaks Efficiently}
\author[Jacob]{Maria Jacob\\ {\small Supervisors: Danica Greetham, Claudia Neves}}


\institute{Mathematics of Planet Earth}
\date{24th May 2017}

\begin{document}

\begin{frame}
  \titlepage
\end{frame}

% Uncomment these lines for an automatically generated outline.
%\begin{frame}{Outline}
%  \tableofcontents
%\end{frame}


\begin{frame}{Introduction}

\begin{itemize}
%\item Motivation
\item Data and Methodology \begin{itemize} \item Forecasting Techniques
\item Validation
\item Extreme Value Theory \end{itemize}
\item Results
\item Future Plans
\end{itemize}

%\vskip 1cm

\end{frame}

\begin{frame}[plain,c]
%\frametitle{A first slide}

\begin{center}
\Huge Data and Methodology
\end{center}

\end{frame}

\begin{frame}{Data}
\begin{itemize}
\item Domestic Smart Meter Data (controlled)
\item Half hourly resolution
\item 503 households
\item 7 weeks (22 weeks)
\end{itemize}
\end{frame}

\begin{frame}{Forecasting}
\begin{itemize}
\item Last Week Forecast (LW)

\begin{center} $\hat{x_n} = x_{n-1}$ \end{center}

\item Same Day Forecast (SD)

\begin{center} $\hat{x}_n = \frac{1}{n-1} \displaystyle \sum_{i=1}^{n-1} x_i$ \end{center}

\item Linear Regression Forecast (LR)

\begin{center} $\hat{x}_n = \displaystyle \sum_{i=1}^{n-1} \beta_i x_i + \epsilon_i$ \end{center}

\item Bayesian Regression Forecast (BR)
\end{itemize}
\end{frame}

\begin{frame}{Error Measures}
\begin{itemize}
\item Mean Absolute Percentage Error (MAPE)

\begin{center} $MAPE^j = \frac{100}{n} \displaystyle \sum_{i=1}^n \left|\frac{x^j_i - \hat{x}^j_i}{x_i}\right|$ \end{center}

\item $p^{th}-$norm error

\begin{center} $E_p^j = \left(\displaystyle \sum_{i=1}^n |\hat{x}^j_i - x^j_i|^p\right)^{\frac{1}{p}}$ \end{center}

where $E_p^j$ is the $p^{th}-$ norm error for household $j$, $\hat{x}^j_i$, $x^j_i$ are the forecast and actual observations for household $j=1, ... , 503$, respectively at time $i = 1, ... , n$.

\end{itemize}
\end{frame}

\begin{frame}{Extreme Value Theory}
\begin{itemize}
\item Suppose that we have $X_1, X_2, ... , X_n, ... $ i.i.d. random variables with common distribution function $F$. The \textbf{sample/block maximum} is defined to be:

\begin{center} $ X_{n,n} = \max\{X_1, ... , X_n\}$ \end{center}

\item The \textbf{right endpoint} is defined as

\begin{center} $x^F := \sup\{x | F(x) < 1\}$ \end{center}

\item The random variables, $X_1, ... , X_n$ can be ordered so that $X_{1,n} \le ... \le X_{n.n}$. Then $X_{k,n} $ for $k \in \mathbb{N}$ is $k^{th}$ upper \textbf{order statistic}.

\end{itemize}
\end{frame}

\begin{frame}{Extreme Value Theorem}

If there exist constants $a_n >0$ and $b_n \in \mathbb{R}$ s.t. \newline

\begin{center}$\displaystyle \lim_{n \rightarrow \infty} \mathbb{P}\left(\frac{X_{n,n} - b_n}{a_n} \le x\right) =  \lim_{n \rightarrow \infty} F^n (a_n x + b_n) = G(x)$ \end{center}

for every continuity of $G$, then $G(x) = G_\gamma(x)$ is given by

\begin{center} $G_\gamma(x) = \exp\{-(1+\gamma x)^{-\frac{1}{\gamma}}\} $ \end{center}

for $1 + \gamma x >0$. $G$ is known as a Generalised Extreme Value (GEV) distribution and $F$ is said to be in the (maximum) domain of attraction of $G_\gamma$, $F \in MDA(G_\gamma)$. \newline


$\gamma$ is known as the \textit{Extreme Value Index} (EVI) or shape parameter.

\end{frame}

\begin{frame}[plain,c]
%\frametitle{A first slide}

\begin{center}
\Huge Results
\end{center}

\end{frame}


\begin{frame}{Patterns}
\begin{figure}
\centering
\includegraphics[width=\textwidth]{16_22_sums.png}
%\setbeamerfont{caption}{series=\normalfont,size=\fontsize{20}{24}}
\caption{Cumulative demand profiles in kilo Watt hours (kWh) for various time horizons.}
\label{fig:sums} 
\end{figure}
\end{frame}

\begin{frame}{Profiles}
\begin{figure}
\centering
\includegraphics[width=\textwidth]{days_sum.png}
\caption{Cumulative daily profiles for each day of the week.}
\label{fig:days} 
\end{figure}
\end{frame}

\begin{frame}{Forecasting}
\begin{figure}
\centering
\includegraphics[scale=0.35]{Forecasts_P3.pdf}
\caption{Daily forecast for one customer for each day of the week.}
\label{fig:days} 
\end{figure}
\end{frame}

\begin{center}
\begin{frame}{Validation}
\begin{tabular}{|c|c|c|}
\hline
Forecast & MAPE (\%) & $E_4$ (kWh) \\ 
\hhline{|=|=|=|}
LW & 84.027 & 4.765 \\
\hline
SD & 81.505 & 3.829\\
\hline
LR & 97.138 & 3.866\\
\hline
BR & 97. 021 & 3.805 \\
\hline
\end{tabular}
\end{frame}
\end{center}

\begin{frame}{EVI using estimators}
\begin{figure}
\begin{center}
\includegraphics[scale=0.45]{GammaEstimates.pdf}
\caption{Estimates for the shape parameter based on the weekly maxima.} \label{fig:gammaEst}
\end{center}
\end{figure}
\end{frame}

\begin{frame}{EVI with Order Statistics }
\begin{figure}
\begin{center}
\includegraphics[scale=0.45]{EVIestimation.pdf}
\caption{Extreme value index estimation using the $k$th largest order statistics.} \label{fig:POTEst}
\end{center}
\end{figure}
\end{frame}

\begin{frame}{Right endpoint}
\begin{figure}
\begin{center}
\includegraphics[scale=0.5]{EndpointEst.pdf}
\caption{Estimates for the right endpoint, based on the weekly maxima.} \label{fig:EndPointEst}
\end{center}
\end{figure}
\end{frame}

\begin{frame}[plain,c]
%\frametitle{A first slide}

\begin{center}
\Huge Conclusion
\end{center}

\end{frame}

\begin{frame}{So far}
\begin{enumerate}
\item Regression techniques performed better.
\item Weekly maxima of electric load are Light tailed.
\item Right Endpoint: 12-13.5 kWh.
\end{enumerate}
\end{frame}

\begin{frame}{What's Next?}
\begin{enumerate}
\item Probabilistic and early biased forecasts
\item Confidence bounds for right endpoint
\item Clustering
\item Heteroscedasticity in the tail observations
\item New estimators for scedasis function
\end{enumerate}
\end{frame}

\begin{frame}[plain,c]
%\frametitle{A first slide}

\begin{center}

\Huge Thank you!

\end{center}

\end{frame}

\begin{frame}{Estimators for $\gamma$}

\begin{itemize} 
\item Maximum Lq-likelihood Estimator (MLq):

\begin{center} $\hat{\gamma} = \arg \displaystyle \max_{\gamma \in \Theta} \sum_{i=1}^n L_q (f(X_i; \gamma)), \quad q >0$ \end{center}
where
\begin{center} $L_q(u) =  \begin{cases} \log u, & \text{if } q= 1 \\ \frac{u^{1-q} -1}{1-q}, & \text{o/w} \end{cases}$ 
\end{center}

and $f$ is the density function of $G$. $q$ is the distortion parameter.

\item Maximum spacing product (MSP)

\end{itemize}
\end{frame}

\begin{frame}{Estimators for $x_F$}

\begin{itemize} 
\item Moment Estimator (M):
\begin{center} $ \hat{\gamma} = 1+ H_n^{(1)} + \frac{1}{2} \left(\frac{\left(H_n^{(1)}\right)^2}{H_n^{(2)}}-1\right)^{-1} $ 
where $ H_n^{(p)} = \frac{1}{k} \displaystyle \sum_{j=1}^k \left(\log(X_{n-j+1,n}) - \log(X_{n-k,n}) \right)^p$
\end{center}
\item Mixed Moment Estimator (MM):

\begin{center} $\hat{\gamma} = \frac{\hat{\phi}_n^k -1}{1+ 2\min(\hat{\phi}_n^k -1, 0)}$  \end{center}

where

\begin{center} $\hat{\phi}_n^k := \frac{M_n^{(1)}(k) - L_n^{(1)}(k)}{\left(L_n^{(1)}(k)\right)^2}, L_n^{(p)} := \frac{1}{k} \displaystyle \sum_{i=1}^k \left( 1- \frac{X_{n-k,n}}{X_{n-i+1,n}}\right)^p\newline M_n^{(p)} := \frac{1}{k} \sum_{i=1}^k \left(1 - \frac{X_{n-i+1,n}}{X_{n-k,n}} \right)^p $ \end{center}


\end{itemize}
\end{frame}



\end{document}

