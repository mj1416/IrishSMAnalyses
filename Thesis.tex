%% ----------------------------------------------------------------
%% Thesis.tex -- MAIN FILE (the one that you compile with LaTeX)
%% ---------------------------------------------------------------- 

% Set up the document
\documentclass[a4paper, 11pt, oneside]{Thesis}  % Use the "Thesis" style, based on the ECS Thesis style by Steve Gunn
%\graphicspath{Figures/}  % Location of the graphics files (set up for graphics to be in PDF format)

\usepackage[a4paper,top=5cm,bottom=0.15cm,left=5.5cm,right=0.1cm,marginparwidth=1cm]{geometry}
\usepackage[english]{babel}
\usepackage[authoryear]{natbib}
\usepackage[utf8x]{inputenc}
\usepackage[T1]{fontenc}
\usepackage[mathscr]{euscript}
\usepackage{makecell}
\usepackage[inline]{enumitem}
\usepackage{amsmath}
\usepackage{bbold}
\usepackage{graphicx}
\usepackage[colorinlistoftodos]{todonotes}
%\usepackage[colorlinks=true, allcolors=blue]{hyperref}

%Useful Commands
\newtheorem{thm}{Theorem}
\newtheorem{lem}[thm]{Lemma}
\newtheorem{rem}[thm]{Remark}
\newtheorem{cor}[thm]{Corollary}
\newtheorem{prop}[thm]{Proposition}
\newtheorem{ex}[thm]{Example}

\usepackage{etoolbox}
\makeatletter
\patchcmd{\backmatter}
{\@mainmatterfalse}
{\@mainmatterfalse\pagenumbering{Roman}}
{}
{}
\makeatother


\newcommand{\id} {\ensuremath{\displaystyle{\mathop {=} ^d}}}


\newcommand{\field}[1]{\mathbb{#1}}
\newcommand{\real}{\ensuremath{{\field{R}}}}
\newcommand{\mc}[1]{{\ensuremath{\mathcal{#1}}}}

\newcommand{\sumab}[2]{\ensuremath{\sum\limits_{#1}^{#2}}}
\newcommand{\intab}[2]{\ensuremath{\int_{#1}^{#2}}}
\newcommand{\intinf}[1]{\ensuremath{\int_{#1}^{\infty}}}
\newcommand{\intunit}{\ensuremath{\int_{0}^{1}}}

\newcommand{\arrowf}[1]{\ensuremath{\displaystyle {\mathop {\longrightarrow}_{#1 \rightarrow \infty}\,}}}
\newcommand{\limit}[1]{\ensuremath{\displaystyle {\lim_{#1 \rightarrow{\infty}}}}}
\newcommand{\suprem}[1]{\ensuremath{\displaystyle {\sup_{#1}}}}
\newcommand{\minarg}[1]{\ensuremath{\displaystyle {\min_{#1}}}}
\newcommand{\argmax}[1]{\ensuremath{\displaystyle {\arg\max_{#1}}}}

% Include any extra LaTeX packages required
%\usepackage[square, numbers, comma, sort&compress]{natbib}  % Use the "Natbib" style for the references in the Bibliography
%\usepackage{verbatim}  % Needed for the "comment" environment to make LaTeX comments
%\usepackage{vector}  % Allows "\bvec{}" and "\buvec{}" for "blackboard" style bold vectors in maths
\hypersetup{urlcolor=black, colorlinks=false}  % Colours hyperlinks in blue, but this can be distracting if there are many links.

\def\myauthor{Maria Jacob}
\def\mytitle{Forecasting Peaks in Household Electric Load Profiles }
%% ----------------------------------------------------------------
\begin{document}
\frontmatter      % Begin Roman style (i, ii, iii, iv...) page numbering

% Set up the Title Page
\title{Forecasting Peaks in Household Electric Load Profiles}
\authors{Maria Jacob}
%\addresses  {\groupname\\\deptname\\\univname}  % Do not change this here, instead these must be set in the "Thesis.cls" file, please look through it instead
\date{$1^{\text{st}}$ September 2017}
\subject{}
\keywords{}

\maketitle
%% ----------------------------------------------------------------

\setstretch{1.3}  % It is better to have smaller font and larger line spacing than the other way round

% Define the page headers using the FancyHdr package and set up for one-sided printing
\fancyhead{}  % Clears all page headers and footers
\rhead{\thepage}  % Sets the right side header to show the page number
\lhead{}  % Clears the left side page header

\pagestyle{fancy}  % Finally, use the "fancy" page style to implement the FancyHdr headers

%% ----------------------------------------------------------------
% Declaration Page required for the Thesis, your institution may give you a different text to place here
\Declaration{

\addtocontents{toc}{\vspace{1em}}  % Add a gap in the Contents, for aesthetics

I, \authornames, declare that this thesis titled, \mytitle and the work presented in it are my own. I confirm that:

\begin{itemize} 
\item[\tiny{$\blacksquare$}] This work was done wholly or mainly while in candidature for a research degree at this University.
 
\item[\tiny{$\blacksquare$}] Where any part of this thesis has previously been submitted for a degree or any other qualification at this University or any other institution, this has been clearly stated.
 
\item[\tiny{$\blacksquare$}] Where I have consulted the published work of others, this is always clearly attributed.
 
\item[\tiny{$\blacksquare$}] Where I have quoted from the work of others, the source is always given. With the exception of such quotations, this thesis is entirely my own work.
 
\item[\tiny{$\blacksquare$}] I have acknowledged all main sources of help.
 
\item[\tiny{$\blacksquare$}] Where the thesis is based on work done by myself jointly with others, I have made clear exactly what was done by others and what I have contributed myself.
\\
\end{itemize}
 
 
Signed:\\
\rule[1em]{25em}{0.5pt}  % This prints a line for the signature
 
Date:\\
\rule[1em]{25em}{0.5pt}  % This prints a line to write the date
}
\clearpage  % Declaration ended, now start a new page

%% ----------------------------------------------------------------
% The "Funny Quote Page"
\pagestyle{empty}  % No headers or footers for the following pages

\null\vfill
% Now comes the "Funny Quote", written in italics
\textit{``Write a funny quote here.''}

\begin{flushright}
If the quote is taken from someone, their name goes here
\end{flushright}

\vfill\vfill\vfill\vfill\vfill\vfill\null
\clearpage  % Funny Quote page ended, start a new page
%% ----------------------------------------------------------------

% The Abstract Page
\addtotoc{Abstract}  % Add the "Abstract" page entry to the Contents
\abstract{
\addtocontents{toc}{\vspace{1em}}  % Add a gap in the Contents, for aesthetics

The Thesis Abstract is written here (and usually kept to just this page). The page is kept centred vertically so can expand into the blank space above the title too\ldots

}

\clearpage  % Abstract ended, start a new page
%% ----------------------------------------------------------------

\setstretch{1.3}  % Reset the line-spacing to 1.3 for body text (if it has changed)

% The Acknowledgements page, for thanking everyone
\acknowledgements{
\addtocontents{toc}{\vspace{1em}}  % Add a gap in the Contents, for aesthetics

The acknowledgements and the people to thank go here, don't forget to include your project advisor\ldots

}
\clearpage  % End of the Acknowledgements
%% ----------------------------------------------------------------

\pagestyle{fancy}  %The page style headers have been "empty" all this time, now use the "fancy" headers as defined before to bring them back


%% ----------------------------------------------------------------
\lhead{\emph{Contents}}  % Set the left side page header to "Contents"
\tableofcontents  % Write out the Table of Contents
 
%% ----------------------------------------------------------------
\lhead{\emph{List of Figures}}  % Set the left side page header to "List if Figures"
\listoffigures  % Write out the List of Figures

%% ----------------------------------------------------------------
\lhead{\emph{List of Tables}}  % Set the left side page header to "List of Tables"
\listoftables  % Write out the List of Tables

%% ----------------------------------------------------------------
\setstretch{1.5}  % Set the line spacing to 1.5, this makes the following tables easier to read
\clearpage  % Start a new page
\lhead{\emph{Abbreviations}}  % Set the left side page header to "Abbreviations"
\listofsymbols{rl}  % Include a list of Abbreviations (a table of two columns)
{
% \textbf{Acronym} & \textbf{W}hat (it) \textbf{S}tands \textbf{F}or \\
\textbf{LCT} & Low Carbon Technologies \\%\textbf{L}ow \textbf{C}arbon \textbf{T}echnologies \\
\textbf{STLF} & Short Term Load Forecasts \\%\textbf{S}hort \textbf{T}erm \textbf{L}oad \textbf{F}orecasts \\
\textbf{PLF} & Probabilistic Load Forecasts \\ %\textbf{P}robabilistic \textbf{L}oad \textbf{F}orecasts \\
\textbf{VSTLF} & Very Short Term Load Forecasts \\
\textbf{MTLF} & Medium Term Load Forecasts \\
\textbf{LTLF}  & Long Term Load Forecasts \\
\textbf{MLR} & Multiple Linear Regression \\
\textbf{SD} & Similar Day \\
\textbf{MAPE} & Mean Absolute Percentage Error \\
\textbf{LW} & Last Week \\
\textbf{AA} & Adjusted Average \\
\textbf{DLM} & Dynamic Linear Model \\
\textbf{EVI} & Extreme Value Index \\
\textbf{BM} & Block Maxima \\
\textbf{POT} & Peak Over Threshold \\
\textbf{HH} & Half Hour \\
\textbf{EVT} & Extreme Value Theory \\
\textbf{GEV} & Generalised Extreme Value \\
\textbf{DF} & Distribution function \\
\textbf{PV} & Photovoltaic \\
\textbf{DNO} & Distribution Network Operators \\
\textbf{WA} & Weighted Average \\
\textbf{BRR} & Bayesian Ridge Regression \\
\textbf{GP} & Generalised Pareto \\
\textbf{QQ} & Quantile-Quantile \\

}
\clearpage
%% ----------------------------------------------------------------
%\clearpage  % Start a new page
%\lhead{\emph{Physical Constants}}  % Set the left side page header to "Physical Constants"
%\listofconstants{lrcl}  % Include a list of Physical Constants (a four column table)
%{
%% Constant Name & Symbol & = & Constant Value (with units) \\
%Speed of Light & $c$ & $=$ & $2.997\ 924\ 58\times10^{8}\ \mbox{ms}^{-\mbox{s}}$ (exact)\\
%
%}

%% ----------------------------------------------------------------
%\clearpage  %Start a new page
%\lhead{\emph{Symbols}}  % Set the left side page header to "Symbols"
%\listofnomenclature{lll}  % Include a list of Symbols (a three column table)
%{
%% symbol & name & unit \\
%$a$ & distance & m \\
%$P$ & power & W (Js$^{-1}$) \\
%& & \\ % Gap to separate the Roman symbols from the Greek
%$\omega$ & angular frequency & rads$^{-1}$ \\
%}
%% ----------------------------------------------------------------
% End of the pre-able, contents and lists of things
% Begin the Dedication page

\setstretch{1.3}  % Return the line spacing back to 1.3

\pagestyle{empty}  % Page style needs to be empty for this page
\dedicatory{For/Dedicated to/To my\ldots}

\addtocontents{toc}{\vspace{2em}}  % Add a gap in the Contents, for aesthetics


%% ----------------------------------------------------------------
\mainmatter	  % Begin normal, numeric (1,2,3...) page numbering
\pagestyle{fancy}  % Return the page headers back to the "fancy" style

% Include the chapters of the thesis, as separate files
% Just uncomment the lines as you write the chapters

\lhead{\emph{Introduction}}  % Set the left side page header to "Abbreviations"

\chapter{Introduction}

Electric load forecasts inform both industrial and societal decision making processes. From energy trading and electricity pricing to demand response and infrastructure maintenance,  electric load forecasts allow distribution network operators (DNOs) and policy makers to prepare for the short and long term future. In order to make informed decisions, the factors influencing electric demand need to be understood, particularly as low carbon technologies (LCT) become more prevalent. 

In order to understand and meet demands effectively, smart grids are being developed in many countries, including the United Kingdom, making high resolution data collected through smart meters accessible. This data allows for better analysis of load, identification of issues and control of electric networks. 

As governments and businesses respond to the threat of climate change and make commitments to move towards technologies powered by electric sources \citep{fuelban}, electric demand will become more volatile. Therefore, though a large body of literature exists in forecasting with smart meter data, there is definite need to revisit the topic and realistically model social behaviour and demographic and meteorological impacts under various future scenarios.

Moreover, rarely though inevitably, businesses and even households may experience power outages. To reduce the impact of these blackouts, households and businesses may consider investing in generators or other electricity storage facilities. However, these technologies are currently very expensive and the purchase of these may need to be justified through rigorous risk analyses. Reducing the uncertainty of blackout occurrences, even by a small amount, may lead to significant financial savings, particularly for businesses. Despite extensive research, the analysis of ``extreme values'' in electric demand have been completely overlooked and it is this analyses that will allow rigorous risk quantification, e.g. of blackouts. The bridging of this gap is the overarching aim of this project. We are particularly interested in short term load forecasting (STLF) at the household level. Thus, this work will focus on producing more accurate forecasts across different demand profiles. To do this, data driven approaches will be combined with extreme value statistics (see chapter \ref{sec:EVT}).

Not only will the combination of electric load forecasts with extreme value statistics allow the quantification of risk from events such as blackouts but it can also help inform if the electric grid needs upgrade, and by how much. It will also allow DNOs to tailor contracts to consumers and provided targeted incentives to reduce peaks in demand. By managing local networks more efficiently, it will be possible to collectively reduce the amount of electricity consumed and thus reducing overall carbon emissions. It is with these industry driven applications that we proceed with this project.

The rest of this chapter will review existing literature on electric load forecasting and outline the objectives of this report. 
% Further, we also want to produce confidence bounds on these limits which can result in financial savings for businesses and help policy makers in decisions regarding infrastructure development.

%Evaluating how LCT and renewable energy sources interact with each other to change and impact the demand at household and substation level will allow energy distributors and policy makers to manage infrastructure development and conduct timely maintenance works.


%As communities and businesses move towards a low carbon future, the demand on the grid will evolve. On the one hand, technologies such as solar panels reduce the demand on the grid but on the other hand the increasing number of electrically powered technology such as such as electric vehicles and charging stations for electric vehicles increase the demand, especially at the low voltage level (households and substations) and may make the load and demand unstable. 

\section{Electric Load Forecasting} \label{subsec:litrev}

As mentioned above, most studies in electric load forecasting in the past century have focused on point load forecasting. However, in the most recent decade researchers have delved into providing probabilistic load forecasts (PLF) as business needs and electricity demand and generation evolve. Forecast horizons for electric load vary from minutes and hours to years and decades. Each forecast horizon has its own application, for example forecasts for up to a day ahead are generated for the purpose of responding to changing demands whereas daily to yearly forecasts may be produced for energy trading and yearly to decadal forecasts to allow for system planning and informing energy policy (fig. \ref{fig:elecfor}).

\begin{figure}
\centering
\includegraphics[width=0.8\textwidth]{elecfor.png}
\caption{Various classifications for electric load forecasts and their applications. The abbreviations are Short Term Load Forecasting (STLF), Very Short Term Load Forecasting (VSTLF), Medium Term Load Forecasting (MTLF) and Long Term Load Forecasting (LTLF). Source: \citet{hong16}}
\label{fig:elecfor} 
\end{figure}

The decision making process in the utility industry relies mostly on expected values so it is no surprise that point load forecasts have been the dominant tool in the past. However, market competition and requirements to integrate renewable technology mean that PLF are increasingly used for system planning and operations. PLF can come in the form of quantiles, intervals and density functions as noted by \citet{hong16} who provided an extensive review of various techniques and methodologies that are used in generating PLF. 

\citet{hong16} referred to techniques as a group of models that fall in the same family. Some of the techniques that were reviewed from the literature included multiple linear regression (MLR), semi-parametric additive models, exponential smoothing models and autoregressive moving average models. To discuss some of these statistical models further, take MLR as an example. MLR uses the load as the dependent variable whereas weather and calendar variables are the independent variables. The algorithm used by \citet{char14} is one example where a MLR technique, specifically a refined semi-parametric model, was used to forecast electricity demand for a specific region in the United States. %The initial code was unrefined and depended only on temperature. The relationship between electricity demand and temperature was found, to a good enough approximation, to be quadratic signifying the use of heating in cooler weather and the use of air conditioning in warmer weather. Additional refinements were added one at a time and their impact on the overall performance of the algorithm and the rationale for why these refinements were introduced were provided. The refinements included:
%\begin{enumerate}
%\item Combining data from multiple weather stations.
%\item Removing outliers.
%\item Treating some national holidays as special cases.
%\end{enumerate}

In contrast to MLR, exponential smoothing models assign weights to past observations that decrease exponentially over time. While these techniques have been quite successful, they have been less readily adopted as a good candidate for real-world short term load forecasting (STLF) \citep{hong16}. These techniques offer the advantage of requiring less data as these models do not usually rely on weather and calendar variables but consequently suffer in cases where the weather is a significant contributor e.g. extremely cold, hot or particularly volatile conditions.

\citet{hong16} went further still and presented some common methodologies, which refer to "general solution framework" that can be implemented with multiple techniques. One methodology that is particularly relevant to this project is the similar day (SD) method; the idea is to find a day(s) in the historical data that is similar to the one being forecasted. For example, a normal weekday such as Thursday may be forecasted as the average of all past Thursdays. \citet{hong16} noted that this method is often used with clustering techniques which combine similar days or similar segments of a day to produce the forecast.

\citet{char14} is yet again a good example of this clustering technique. The data were clustered into various zones depending on geography, into two seasons, into 24 hours of the day and either weekday or weekend. Thus, the authors treat load to be categorically different for a weekday and weekend and take geography and seasonality to be impactful. This is an example of a study where data are clustered before the forecast is produced. In contrast \cite{dan14} uses the forecast, specifically an error metric to cluster roughly 600 households into three categories, one where the forecasting skill is poor, the second where the skill is good after being adjusted and the third where it is good overall. The forecasting skill here refers to how good the forecast is in comparison to the observed load.

Another algorithm of interest is the one used in \citet{douglas98} which used Bayesian estimation together with a dynamic linear model (DLM) to create short-term forecasts for Oklahoma City. This study was ultimately interested in quantifying the impact of imperfect weather data on forecasts. This however is not what is relevant for this project instead it is the use of Bayesian estimation. %(described below). 
The method proposed requires historical load data, historical temperature data and temperature forecast data. The temperature data, acquired through the National Weather Service, traditionally only contains the average, highest, and lowest temperature for any given day. Thus, while two models were proposed the hourly model, which requires hourly temperature inputs, was not used. The model that was used was called the daily peak model in which ``peak load is forecasted, and then a typical load profile is linearly scaled to generate the required load profile'' \citep{douglas98}. It was noted in the study that, for days when drastic weather changes or anomalies are not occurring, a ``typical'' day is a reasonable representative of the load for that day and thus can be used to forecast the load reasonably accurately. The authors do not quantify ``typical'' however they do say that it is generated from the historical load data. Thus it is reasonable to take ``typical'' to be the mean or median. \cite{douglas98} does not offer alternative strategies to deal with days when there are drastic weather changes or anomalies. %This model uses peak temperature, average temperature, and temperature at hour 24 from the previous day and peak and average temperature forecast for the relevant days as the explanatory variables in a linear regression model. Since the process is set up so that the load for the typical day passes both the peak load forecast and the load at hour 24 of the previous day, a typical load for the hour 24 of the previous day is required and subsequently a typical week must be generated to support the execution of this procedure. 
%The Bayesian estimation procedure is described below.
%\begin{enumerate}
%\item Let us set up the problem first. A state space is given in equation \ref{eq:statespace}, where $\theta_t$ is the state vector which is time-dependent with dimensions $(k \times 1)$, $G$ is the state evolution matrix with dimension $(k \times k)$, $Y_t$ is the scalar output, $F_t$ is the regression vector with dimensions $(k \times 1)$, $V_t$ is the scalar variance and $W_t$ is the covariance matrix with dimensions $(k \times k)$.) $\mathscr{N}(a,b)$ denotes the normal distribution with mean $a$ and variance $b$ and $(\cdot)^T$ denotes the transpose.
%\begin{align} \label{eq:statespace}
%\begin{split}
%\theta_t &= G\theta_{t-1} + \omega_t, \quad \omega \sim \mathscr{N}(0,W_t) \\
%Y_t &= F_t\theta_{t-1} + \nu_t, \quad \nu \sim \mathscr{N}(0,V_t)
%\end{split}
%\end{align}
%%
%\item The \textit{a posteriori} distribution for time $t-1$ ($t$ is in days) and \textit{a priori} distribution for time t is shown in equation \ref{eq:pos_pri}. In these set of equations, $D_{t-1}$ is all the information that is available at time $t-1$, $C_{t-1}$ is the posterior convariance matrix, $m_{t-1}$ is the posterior state mean vector. $a_t$ and $R_t$ are the prior mean and covariance at time $t$. $n_{t-1}$ is the degrees of freedom of the student-t test, which is denoted by $T$ and $S_t$ is an estimate of the scalar output variance at time t.
%%
%\begin{align} \label{eq:pos_pri}
%\begin{split}
%(\theta_{t-1}|D_{t-1}) &\sim T_{n_{t-1}}(m_{t-1},C_{t-1}) \\
%(\theta_{t-1}| D_t) &\sim T_{n_{t-1}}(a_t,R_t)
%\end{split}
%\end{align}
%%
%where
%%
%\begin{equation*}
%a_t = Gm_{t-1}, \qquad R_t = \delta^{-1} \cdot GC_{t-1}G^T
%\end{equation*}
%%
%\item The forecast, for a day ahead, distribution then follows equation \ref{eq:bayforecast}. In addition to those variables defined above, $f_t$ and $Q_t$ are the mean and variance of the scalar forecast distribution.
%%
%\begin{align} \label{eq:bayforecast}
%\begin{split}
%(Y_t | D_{t-1}) \sim T_{n_{t-1}}(f_t,Q_t) \\
%\end{split}
%\end{align}
%%
%where
%%
%\begin{equation*}
%f_t = F^T_t a_t, \qquad Q_t = F_t R_t F_t^T + S_t
%\end{equation*}
%%
%\item Finally, the $a$ $posteriori$ distribution for time $t$ is calculated in accordance with equation \ref{eq:pos_update}, where $m_t$ and $C_t$ now are the posterior mean and covariance. The adaptive factor, $A_t$, in practice is often replaced by a scalar, namely $\delta$ which is then called a discount factor. A high $\delta$ is indicative of the relatively high importance of past load. The best value of $\delta$ was found to be 0.93 \citep{douglas98}.
%%
%\begin{align} \label{eq:pos_update}
%(\theta_t|D_t) \sim N(m_t,C_t)
%\end{align}
%%
%where
%%
%\begin{align*}
%\begin{split}
%m_t &= a_t + A_te_,t \\
%C_t &= U_t R_t U_t^T + A_t S_t A_t^T, \\
%U_t &= (I - A_tF_t^T), \\ 
%e_t &= Y_t - f_t \\
%A_t &= R_tF_tQ_t^{-1}, \\ 
%n_t &= n_{t-1} + 1\\
%d_t &= d_{t-1} + e^2S_tQ_t^{-1}
%\end{split}
%\end{align*}
%%
%\item Where the forecast is required for the $k^{\text{th}}$ day ahead, then the forecast distribution is given as shown in equation \ref{eq:kbayforecast} where all the variables are as before except $Y_{t+k}$ is the $k^{t\text{th}}$ step ahead forecast distribution.
%%
%\begin{align} \label{eq:kbayforecast}
%(Y_{t+k}|D_t) \sim T_{n_{t-1}}(f_t(k),Q_t(k))
%\end{align}
%%
%where
%%
%\begin{align*}
%\begin{split}
%f_t(k) &= F_t^T,a_t(k) \\
%Q_t(k) &= F_t R_t(k) F_t^T + S_t \\
%a_t(k) &= Gm_t(k-1) \\
%R_t(k) &= \delta^{-1}\cdot GR_t(k-1)G^T
%\end{split}
%\end{align*}
%%
%\end{enumerate}
%
%To generate a daily forecast, a set of 2 linear equations (eq. \ref{eq:simul_bay}) are solved. These equation depend on the $\hat{l}(p)$, the peak load estimate at hour $p$ with $p=0$ corresponding to the load estimate at hour 24 of the previous day, and on $l_t(p)$ being the typical day peak load at hour $p$ again with $p=0$ corresponding to hour 24 of the previous day.
%
%\begin{align} \label{eq:simul_bay}
%\begin{split}
%\hat{l}(p) &= a + bl_t(p) \\
%\hat{l}(0) &= a + bl_t(0)
%\end{split}
%\end{align}
%
%\noindent The solution then looks like equation \ref{eq:simul_bay_sol}.
%
%\begin{align} \label{eq:simul_bay_sol}
%\begin{split}
%a &= \frac{l_t(p)\hat{l}(0) - l_t(0)\hat{l}(p)}{l_t(p) - l_t(0)} \\
%b &= \frac{\hat{l}(p) - \hat{l}(0)}{l_t(p) - l_t(0)}
%\end{split}
%\end{align}
%
%\noindent Thus the typical day can then be (linearly) scaled according to equation \ref{eq:scale}.
%
%\begin{equation} \label{eq:scale}
%\hat{l}(i) = a + bl_t(i), \qquad \forall i= 1,...,24
%\end{equation}



\section{Forecast Uncertainty}

No forecast can be adopted into real-world practice without assessing the quality of it. There are many ways to make these assessments or more strictly many error metric that can be used. %Since some forecasting techniques and methods have been discussed it is also valuable to discuss how to assess the goodness of a forecast. 
Some conventional error metrics for load forecasts are mean absolute percentage error (MAPE) and mean absolute error \citep{hong16}. These error metric are reasonably simple and transparent and thus quite favourable. However, as noted by \cite{dan14}, for STLF a peaky forecast is more desirable and realistic than a flat forecast but error metrics such as mean square errors and MAPE unjustly penalise peaky forecasts and can often quantify the flat forecast to be better. This is because the peaky forecast is penalised twice: once for the peak not occurring at the exact same point where the observed peak occurs and again for the peak occurring at some point slightly shifted from where the observed peak occurs. A flat-forecast does not incur this double penalty. \citet{dan14}, therefore, developed an adjusted error metric that penalises less so a forecast which predicts a peak that is slightly shifted than a forecast where the peak is not predicted at all. The authors also test out three different forecasting methods:
\begin{enumerate}
\item flat forecast: a horizontal line determined from an average of past data.
\item Last week (LW) forecast: The same usage as in the previous, e.g. forecast for Thursday is predicted to be the same as the Thursday from the previous week.
\item Averaged Adjustment (AA) Forecast: takes into account both a weighted historic average and a baseline usage, where permutations in time are allowed.
\end{enumerate}

Using both the new error measure, relative to the typical error measure, as well as the mean displacements in peaks, \cite{dan14} show that the flat forecast is not a good forecasting method for high resolution smart meter data and that in general the best technique out of the three is AA with LW performing relatively well too. Thus there is evidence to suggest that while some forecasts are good at forecasting electric load at national ad regional load, the same forecast are unsuitable for forecasting load of households and low voltage networks. Thus we need to investigate the quality of existing forecasts in the context of household load and/or device new ones to ensure that forecasts enable sound decisions to be made.


\section{Objectives} \label{subsec:objectives}

There are two main objectives in this project. The first objective is to create benchmark point load forecasts and error measures. This will be done by implementing some of the forecasts discussed before and validating them with existing error measures. The second objective is to set up the extreme value framework for electric load. In order to do this, we will conduct some preliminary exploratory statistical analysis to understand the nature of ``extreme'' values . Further to this we will also estimate the extreme value index (EVI) and the right endpoint using the Block Maxima (BM) method. Finally we will use the Peak Over Threshold (POT) method to estimate the tail relative risk in the case where the data are not identically distributed i.e for heteroscedastic extremes under the assumption that the EVI remains constant in time. The results of this study will answer some practical industry-based questions as well as being strongly grounded in the theoretical framework.


\section{Outline}

Thus far we have reviewed some of the literature in electric load forecasting and outlined the objectives of the project. Chapter 2 follows next wherein we familiarise ourselves with the data. In chapter 3 we tackle some existing and modified forecasts and discuss the goodness of each. Chapter 4 will review the mathematical framework of extreme value statistics and discuss the results as applied to electric load forecasting. Finally we will conclude the results of this work in chapter 5.
 % Introduction
\clearpage

\lhead{\emph{Data}}  % Set the left side page header to "Abbreviations"

\chapter{Data} \label{sec:results}

The Irish Social Science Data Archive has made smart meter data publicly available for about 5000 households in Ireland \citep{issda}. The Smart Metering Project was launched in 2007 with the intention of understanding consumer behaviour with regard to the influence of smart meter technology. The data used here are for the households that were used as the controls in the trials. There are 7 weeks of data at half hourly resolution, where the weeks are labelled from 16 to 22 (inclusive). The half hours are labelled from 1 to 48 where 1 is understood to correspond to midnight. Additionally days are also numbered. In this case, the numbering starts at 593 which is understood to be the 16th of August 2010 and the numbering ends at 641. From this the days of the weeks, ranging from 1 to 7 where 1 is Monday and 7 is Sunday, has been deduced. Being equipped with this knowledge, we explore some basic properties of the data.

\section{Profiles and Trends}
\label{subsec:basic} 
This section presents some basic visualisation and discussion of the general nature of electricity demand. First consider the histogram of the measurements shown in figure \ref{fig:hist}. The $75^{\text{th}}$ percentile of this data is 0.5 kWh so clearly most households use less than 0.5 kWh at any given time, however there are households which record almost as high as 12 kWh. These high consumers are most likely operating a small business from home and/or may have multiple large appliances and electric vehicles in their homes.  While this is a plausible explanation in general, for this data set this last part does not seem to be the case as an electric vehicle recharging is a constant electric demand which lasts for several hours. Such sustained demand was not observed in the data.

\begin{figure}
\centering
\includegraphics[width=\textwidth]{usage_histogram.png}
\caption{histogram of the HH smart meter readings for all 503 households.}
\label{fig:hist} 
\end{figure}

Where figure \ref{fig:hist} told us about half hourly (HH) demand, figure \ref{fig:sums} gives some general profiles. These four plots show the total/cumulative pattern of electricity demand. The top left image, shows the dip in usage overnight, the increase for breakfast which stabilises during typical working hours and rises again for dinner . These are as expected. Similarly the bottom left image shows a recurring pattern indicating that there are specific days in the week where usage is relatively high and other days where it is relatively low. This is further confirmed by the image on the bottom right and further tells us that in total, Fridays are the least heavy day of the week whereas Weekends are typically the most heavy. The image on the top right shows a rise in demand starting in week 18, which is around the beginning of September, aligning with the start of the academic year for all primary and some secondary schools in Ireland. This explains why the jump in data occurs as the weeks preceding are weeks when many families may travel abroad and thus have little usage.

\begin{figure}
\centering
\includegraphics[width=\textwidth]{16_22_sums.png}
\caption{Cumulative demand profiles in kilo Watt hours (kWh) for various time horizons.}
\label{fig:sums} 
\end{figure}

It is also valuable to see how the top left profile changes for each day of the week (fig. \ref{fig:days}). From this image it is clear that there are differences between weekdays and weekend. Clearly, the breakfast peak is delayed on weekends with no categorical differences in the evening peaks between weekends and weekdays. This is useful for clustering within forecasting.

\begin{figure}
\centering
\includegraphics[width=\textwidth]{days_sum.png}
\caption{Total daily profiles for each day of the week.}
\label{fig:days} 
\end{figure}

One way to see if there are ``extreme'' households is to breakdown the daily consumption by each household. This is shown in figure \ref{fig:totes}. The various colours show the various households though it should be noted there is not a unique colour for each household. It is noteworthy that there is one house (coloured in light blue) that consistently appears to be using the most amount of energy per day. It could be that this house has consistently high demand due to a small business that its occupants are operating from home.

\begin{figure}
\centering
\includegraphics[width=\textwidth]{tot_daily_per_customer.png}
\caption{Total daily usage per household for each day in the data.}
\label{fig:totes}
\end{figure}

One last thing to be considered in this section is the auto-correlation. It is reasonable to suspect that the demands of households are correlated to its past demand and that future weekdays will be like past weekdays and future weekends will be like past weekends and similarly that today's demand is much like yesterday's demand (if yesterday and today are both weekends or both weekdays). Indeed all of of our forecasts rely on this property so it is useful to verify it.  Figure \ref{fig:ty_colour} shows the linear relationship between the total daily demand of each house, again colour coordinated, on day d and day d-1 at the same HH and clearly there is some linear trend here. To see how far back this relationship holds, an autocorrelation function for one day (fig. \ref{fig:acf_day}) is provided. As it can be seen that there is some symmetry and while it is not shown here there is also periodicity throughout the data set though with decreasing autocorrelation. The autocorrelation function plotted in figure \ref{fig:acf_day} is an arithmetic mean of all customers at each HH.

\begin{figure}
\centering
\includegraphics[width=0.9\textwidth]{autocorr_t_t-1.png}
\caption{Daily total electric load for day d against day d-1.}
\label{fig:ty_colour} 
\end{figure}

\begin{figure}
\centering
\includegraphics[width=0.9\textwidth]{r_autocorr_day.png}
\caption{Auto-correlation function for 1 day. Lag is measured in HH}
\label{fig:acf_day} 
\end{figure}




 % Data Description and visualisation
\clearpage

\lhead{\emph{Forecasting Electric Load}}  % Set the left side page header to "Abbreviations"

\chapter{Electric Load Forecasts}
We reviewed many of the existing forecasting methods and techniques in section \ref{subsec:litrev} and we've discussed the value of having good forecasts with some commentary on how the quality of a forecast may be judged. In this chapter, various forecasts will be generated and their goodness discussed both qualitatively and quantitatively. Due to the strong periodic and symmetric autocorrelation (fig. \ref{fig:acf_day}), we can confidently use many of the common forecasting methods in the literature and build on those. Recall that we have 7 weeks of data, where we treat weeks 16-21 (inclusive) as past data and forecast week 22.

\section{Forecasts} \label{subsec:forecasts}

\subsection{Similar Day Forecast}

The first one considered here is the Similar Day (SD) method which forms the foundation of most electric load forecasting methods and is quite successful in itself as we'll see later. Therefore, we will use this forecast as the benchmark. The SD forecast is generated as follows:
\begin{enumerate}
\item The forecast at a given half hour for a specific day of the week is taken to be the arithmetic mean of all past observations which exist at the same half hour for the same day of the week. For example, the forecast for 9 am on Monday of week 22 (the last week in the 7 week data) will be the arithmetic mean of measurements taken at 9 am on Mondays of weeks 16 to 21.
\item This is repeated for each half hour for each day of the week for each household.
\item Thus a week long forecast is generated is at half hourly resolution.
\end{enumerate}

A sample forecast showing also the measured load for one customer is provided in figure \ref{fig:SDforecast}. From this, we can see that though the peaks are largely underestimated, they occur at roughly the right time, specifically the breakfast hours for the weekdays. Similarly, we see peak in forecast for the evening time, they are not underestimated but much more smooth than the observed load. This is because the timing for breakfast is more strictly constrained by jobs and other external deadlines. Additionally, it maybe argued that the underestimation of peaks is an effect of reduced load in the first weeks of the past data (recall fig. \ref{fig:sums}), though more realistically it is due to the smoothing effect from averaging.

\begin{figure}
\centering
\includegraphics[scale=0.5]{forecast_SD_P1.pdf}
\caption{SD forecast (for each day of the week) and the observations for customer 1.}
\label{fig:SDforecast} 
\end{figure}

\subsection{Last Week Forecast}

Another common but simple forecasting technique is the Last Week (LW) Forecast. It can be seen as a special case of the SD forecast where instead of using all historical data only the most recent week is used. Even before we test this method, it is probable that this is not a good forecasting method as relevant data, if available, may go unused and does not account for week to week variability. However, where data is not available, this may be the only option and in some cases may be a good enough option i.e for households which are very predictable.

As before, for the same customer, the LW forecast versus the observations are shown in figure \ref{fig:LW_forecast_P1}. It is clear that at least in the sample case LW is also not very well suited to representing the peaks accurately and is particularly vulnerable due to the week to week volatility in load. However, it can be seen that there are cases where, even though the magnitude of the peak is not correct, the peaks are being forecasted at roughly the right time. With this in mind we can combine the ideas from the SD and LW forecasts to generate an adjusted average forecast.

\begin{figure}
\includegraphics[scale=0.5]{forecast_LW_P1.pdf}
\caption{LW forecast (for each day of the week) for customer 1.}
\label{fig:LW_forecast_P1} 
\end{figure}

\subsection{Adjusted Average Forecast}

\cite{dan14} introduced an adjusted average (AA) forecast. The idea is that we start with a last week forecast and iteratively update it to included information from previous weeks with the most recent week having the most influence on the end forecast. The iterative update involved a local, restricted permutation in time which minimises some cost function. The algorithm as outlined by \cite{dan14} is as follows:

The forecast, as described by \cite{dan14}, forecasts each day of the week individually though analogously.
\begin{enumerate}[label=\roman*)]
\item Let $d$ be the day of the week being forecasted $(d=1,...,7)$ and suppose that there are $N$ daily usage profiles (i.e. historical data for day $d$ exists for the most recent $N$ weeks) at half hourly resolution. For $k = 1, ..., N$, the daily usage profile is denoted by $\boldsymbol{G}^{(k)} = (g_1^{(k)}, ... , g_{48}^{(k)})^T$. Note the convention that $\boldsymbol{G}^{(1)}$ is the profile for day $d$ in the most recent week whereas $\boldsymbol{G}^{(N)}$ is of the earliest week.
\item A base profile, $\boldsymbol{F}^{(1)} = \left(f_1^{(1)}, ... , f_{48}^{(1)} \right)^T$. Each $f_i^{(1)}$ is defined by the median of past load of the corresponding half hour i.e $ \forall \quad i \in (1, ..., 48), \quad f_i^{(1)} = \text{median}(g_i^{(1)}, ..., g_i^{(N)})$.
\item This baseline is updated iteratively to get the final forecast, $\boldsymbol{F}^{(N)}$, in the following way. Suppose, at iteration $k$, we have the $\boldsymbol{F}^{(k)}$ for $1 \le k \le N-1$, then $\boldsymbol{F}^{(k+1)}$ is obtained by setting $\boldsymbol{F}^{(k+1)} = \frac{1}{k+1} \left( \boldsymbol{\hat{G}}^{(k)} + k \boldsymbol{F}^{(k)}\right)$, where $\boldsymbol{\hat{G}}^{(k)} = \hat{P}\boldsymbol{G}^{(k)}$ with $\hat{P} \in  \mathscr{P}$ being a permutation matrix s.t. $||\hat{P}\boldsymbol{G}^{(k)} - \boldsymbol{F}^{(k)}||_4 = \displaystyle \min_{P \in \mathscr{P}}||P\boldsymbol{G}^{(k)} - \boldsymbol{F}^{(k)}||_4 $.
\item $\mathscr{P}$ is the set of restricted permutations i.e, for a chosen deformation limit $\omega$, the load at half hour $i$ can be moved to some half hour $j$ if $|i-j| \le \omega$. In \cite{dan14} and this report, $\omega=3$.
\item Thus one can see that the final forecast is given by $\boldsymbol{F}^{(N)} = \frac{1}{N+1}\left(\displaystyle \sum_{k=1}^n \boldsymbol{\hat{G}}^{(k)} + \boldsymbol{F}^{(1)} \right)$.
\end{enumerate}

The Hungarian algorithm is used to solve the minimisation problem above, which can be thought of as an optimisation problem. The Hungarian algorithm uses a cost matrix and thus for this implementation we must also specify a cost matrix. Recall that we have a deformation limit, $\omega$ which we set to 3. This comes into play as for any index $i = 1, ..., 48$, which relates to half hour, $(i,j)^{th}$ element is set to be the cost if $|i-j|\le\omega$ or set to be infinity if $|i-j| > \omega$. Thus at iteration $k$, the cost matrix looks like: \newline

\centerline{$\begin{bmatrix}
    |g_1^{(k)} - f_1^{(k)}| & |g_2^{(k)} - f_1^{(k)}| & |g_3^{(k)} - f_1^{(k)}| & |g_4^{(k)} -f_1^{(k)}| & \infty  & \infty& \infty & \infty & \dots & \infty \\
    |g_1^{(k)} - f_2^{(k)}| & |g_2^{(k)} - f_2^{(k)}| & |g_3^{(k)} - f_2^{(k)}| &  |g_4^{(k)} - f_2^{(k)}| & |g_5^{(k)} - f_2^{(k)}| & \infty & \infty &\infty & \dots & \infty\\
    |g_1^{(k)} - f_3^{(k)}| & |g_2^{(k)} - f_3^{(k)}| & |g_3^{(k)} - f_3^{(k)}| &  |g_4^{(k)} - f_3^{(k)}|& |g_5^{(k)}- f_3^{(k)}| & |g_6^{(k)}-f_3^{(k)}| & \infty & \infty & \dots  & \infty\\
    |g_1^{(k)}-f_4^{(k)}| & |g_2^{(k)} - f_4^{(k)}| & |g_3^{(k)} - f_4^{(k)}| & |g_4^{(k)} - f_4^{(k)}| &  |g_5^{(k)} - f_4^{(k)}| & |g_6^{(k)}- f_4^{(k)}|& |g_7^{(k)} - f_4^{(k)}| &\infty &\dots & \infty\\
    \infty & |g_2^{(k)} - f_5^{(k)}| & |g_3^{(k)} - f_5^{(k)}| & |g_4^{(k)} - f_5^{(k)}| &  |g_5^{(k)} - f_5^{(k)}|& |g_6^{(k)}- f_5^{(k)}|& |g_7^{(k)} - f_5^{(k)}| &|g_8^{(k)} - f_5^{(k)}| &\dots & \infty\\
     &  & \dots &  & &  & &  & \\
\end{bmatrix}$}

This algorithm is intuitively quite suited for forecasting as it gives a peaky forecast and at least for the sample forecast (fig. \ref{fig:AA_forecast_P1}) we can see that this is the case. Obviously not all peaks are represented well however from looking at this one customer, it seems that more peaks are being forecasted and roughly the correct times. The magnitudes are still not correct however in many cases they are closer to the observed than the SD forecast. We will quantify this further in section \ref{subsec:errs}. This should be the best forecast at least when the error measure used is the adjusted error measure.

\begin{figure}
\centering
\includegraphics[scale=0.5]{forecast_AA_P1.pdf}
\caption{AA forecast (for each day of the week) for customer 1 where the solid black line is the observed load, the broken black line is the SD forecast and broken blue line is the AA forecast.}
\label{fig:AA_forecast_P1} 
\end{figure}

\subsection{Weighted Average Forecast}

We can now move on to some other modifications of the SD forecast. The SD forecast weighted all past data equally and didn't use any window as in the AA forecast even though it is reasonable to think that a window of data from past weeks would be appropriate. Instead of guessing the weights we can calculate the weights using linear regression. As \cite{hong16} noted there is a misconception that linear regression can only be used in cases where there is a linear relationship between the independent and dependent variable. In fact linear regression is appropriate in applications where the model depends linearly on the unknown parameters, $\boldsymbol \beta$, (equation \ref{eq:lin_reg}) but the relationship between the features need not be linear.

Given a data set $\{y_i, x_{i1}, ... , x_{ip}\}_{i=1}^n$, linear regression, in vector form, can be represented as shown in equation \ref{eq:lin_reg}.

\begin{equation} \label{eq:lin_reg}
\textbf{y} = X \boldsymbol \beta +\boldsymbol \epsilon
\end{equation}
where $\textbf{y} = (y_1, y_2, ... , y_n)^T $, $X =  (\textbf{x}_1^T, \textbf{x}_2^T,, ... ,\textbf{x}_n^T)^T$,  $\textbf{x}_i^T, = (x_{i1}, ... , x_{ip})$ for $i = 1, ... , n$, $\boldsymbol \beta = (\beta_1 , ... , \beta_p)$ and $\boldsymbol \epsilon = (\epsilon_1, ... , \epsilon_n)$ and $( \cdot )^T$ denotes transpose. The $\boldsymbol \epsilon$ is a disturbance or error variable which adds noise to the model and is commonly assumed to be normally distributed with zero mean and variance $\sigma^2$.

Thus, in the context of electric load, we can use linear regression to predict each load at each half hour by modelling the relationship between the half hour in question and available observations that we deem relevant. We can inform ourselves of what is relevant by utilising the correlations (recall fig. \ref{fig:acf_day}). Thus by running 48 linear regressions, a one day forecast can be generated. The specifics of the implementation are as follows:
\begin{enumerate}
\item Using the python's SciKit package, linear regression is done on variable $X$ and $Y$, where $X$ is the matrix of features and $Y$ is the target variable.
%\item As mentioned before, the data set contains 7 weeks of data. These weeks range from 16 to 22. Week 22 is the week being forecasted and weeks 16 to 21 are considered "past" data which will form part of the feature matrix. 
\item The implementation contains two steps: \begin{enumerate} \item The first step is to find/learn the parameters $\boldsymbol \beta$. To do this both $X$ and $Y$ are required. In this first step, $Y$ is a vector which contains information about the half hour in question for each household for the most recent "past" week i.e week 21 whereas matrix $X$ contains data for the same half hour as well a lag 1 hour in each direction from weeks 16 to 20 (inclusive). Thus by using ordinary least squares, $\boldsymbol \beta$ is found by fitting $X$ to $Y$.
\item In the second step, the forecast is created. The python implementation requires the feature matrix in the fitting phase and the forecasting phase to have the same dimension thus $X$ has to be adjusted. Since we are now forecasting week 22, we use the same half hour and lags but are taken from weeks 17 to 21 to ensure that the new $X$ has the same dimensions as the old $X$. The values predicted by the model are the considered to be the forecast for the half hour in question for week 22. \end{enumerate}
\item The forecast that is produced using the above step is for the prescribed half hour of the prescribed day of the week. Thus, both the desired half hour and the day of the week are required inputs of the implementation.
\item The daily forecast for day $d$ is then produced by repeating the two steps for each half hour.
\item Then acquiring the weekly forecast, just means running the daily forecast for each day of the week.
\end{enumerate}

In this way it is possible to think of the SD forecast as a special case of the linear regression where the coefficients/parameters of the lag are zero in the SD forecast and the others are equal to 1 regardless of when measurements were made. This forecast will be referred to as the Weighted Average (WA) forecast, a sample of which along with the observations is shown in figure \ref{fig:LR_forecast_P1}. Given that many of the existing algorithms in the literature are some based around regression techniques, it is reasonable to expect that this is better than the LW forecast and also potentially the SD forecast given that it uses more data and the weights have been calculated rather than just given equal weighting. However, since the load profile is irregular, the peaks are likely to get smoothed in the process and may not be ideal for this context. This can explicitly be seen in figure in the sample forecasts given in figure \ref{fig:LR_forecast_P1} but we will quantify this more concretely as we proceed. %and \ref{fig:forecasts_P1}. 
\begin{figure}
\centering
\includegraphics[scale=0.5]{forecast_LR_P1.pdf}
\caption{Weighted Average forecast (for each day of the week) for customer 1.}
\label{fig:LR_forecast_P1} 
\end{figure}

\subsection{Bayesian Ridge Regression Forecast} % \label{subsubsec:BR}

Since we have an example of at least one paper \citep{douglas98} where the Bayesian framework has been applied, we can apply it too. Python's SciKit package provides an in built option with ridge regression, thus we call the following algorithm Bayesian Ridge Regression (BRR) forecast. Before looking at the results, let's look at the basic background of ridge regression. Ridge regression is the most common method of regularisation of ill-posed problems in statistics. A problem is said to be ill-posed if it has no solution or multiple solutions. Supposed we wish to find an $\boldsymbol{x}$ such that $A\boldsymbol{x}=\boldsymbol{b}$, where $A$ is a matrix and $\boldsymbol{x}$ and $\boldsymbol{b}$ are vectors. The ordinary least squares estimation solution would be a gained by a minimisation of $||A\boldsymbol{x} - \boldsymbol{b}||_2$ however for the ill-posed problem, this solution may be over-fitted or under-fitted. To give preference to a solution with desirable properties, the relgularisation term $||\Gamma\boldsymbol{x}||_2$ so that the minimisation is of $||A\boldsymbol{x} - \boldsymbol{b}||_2 + ||\Gamma\boldsymbol{x}||_2$, which gives the solution $\hat{\boldsymbol{x}} = \left(A^T A + \Gamma^T \Gamma \right)^{-1} A^T \boldsymbol{b}$. For the Bayesian interpretation,simplistically this regularised solution is the most probable solution given the data and the prior distribution for $\boldsymbol{x}$ according to Bayes' Theorem. The Python implementation is very much like the weighted average, where the coefficients are identified using the weeks 16 to 20 and the forecast for week 22 is generated using weeks 17 to 21.

Lastly we visualise a sample of BRR forecast along with the observations for one customer (fig. \ref{fig:BR_forecast_P1}). The forecast, at least for this customer, is qualitatively very similar to the WA forecast as we would expect very smooth and may thus be unsuitable for the context of load forecasts. It is also noteworthy that though this data set has no bank holidays and change of seasons, it does have a fundamental shift in behaviour due to schools opening half way through when measurements are taken. Perhaps these algorithms would be more successful if the forecasted week were also very similar to all the historical data being used to generate the forecast. We will quantitatively verify all the above forecasts in the next section.

\begin{figure}
\centering
\includegraphics[scale=0.5]{forecast_BR_P1.pdf}
\caption{Bayesian Ridge Regression forecast (for each day of the week) for customer 1.}
\label{fig:BR_forecast_P1} 
\end{figure}
\clearpage


\section{Forecast Validation} \label{subsec:errs}
As was discussed before in \ref{subsec:litrev}, there are many ways to judge a forecast. There is no absolute way of measuring the goodness of a forecast and depending on the error measure used, the conclusion may be qualitatively different (as will be seen later). Though there are no correct measures, some measures may be more appropriate or suitable than others for the application at hand. In this section we will consider two error measures, one which is quite commonly used and another which was developed by \cite{dan14} with electric load in mind.

The first one is the common $p-$norm error described in equation \ref{eq:err_p}, where $ p > 1 $.

\begin{equation}\label{eq:err_p}
E_p \equiv ||\boldsymbol{f} - \boldsymbol{a}||_p := \left( \sum_{i=1}^{n} |f_i - a_i |^p\right)^{\frac{1}{p}}
\end{equation}

where $\boldsymbol{f}$ is the forecast that is $n$ half hours long and $\boldsymbol{a}$ is the observed measurement. This gives the $p-$norm error for each household. Commonly, $p=2$ is taken, which translates to mean square root error however in this report we will take $p=4$ as has been done \citet{dan14} \todo{check this} so as to exaggerate larger differences. This was calculated for each household and is shown in figure \ref{fig:m4e_all}. The black line is the error at each household whereas the flat blue line is the average of all the households. These average has been tabulated in table \ref{tab:errs}.

\begin{figure}
\centering
\includegraphics[scale=0.5]{Errors_E_4.pdf}
\caption{\label{fig:m4e_all} $4^{th}$-norm error for each forecast.}
\end{figure}


The $4^{\textbf{th}}-$norm error provides a good benchmark for an error measure, however as was discussed before a point error metric such as this one falls prey to the double penalty effect. In order to reduce this effect we can do a restricted permutation locally in time by using the error measure introduced in \citet{dan14}. The mathematical set up for this error measure is as follows:
\begin{itemize}
\item $\hat{E}_p^\omega = \displaystyle{\min_{P \in \mathscr{P}}||P\textbf{f}-\textbf{x}||_p}$, where \textbf{f} is the forecast and \textbf{x} are the observations and where $\mathscr{P}$ represents the set of restricted permutations i.e. we allow the forecast to be matched to an observation within some window. The window was chosen to be 2 and a half hours, i.e. $\omega = 3$, as in \cite{dan14}. It should be noted that the value of $\omega$ need not be the same for the AA algorithm and the adjusted error measure.
\item The solution to above minimisation problem can be found using the Hungarian algorithm as for the AA forecast. The cost matrix for the error measure is then given as


\centerline{$\begin{bmatrix}
    |f_1 - x_1| & |f_2 - x_1| & |f_3 - x_1| & |f_4 - x_1|  & \infty& \infty& \infty &\infty & \dots & \infty \\
    |f_1 - x_2| & |f_2 - x_2| & |f_3 - x_2| &  |f_4 - x_2| & |f_5 - x_2| & \infty  &\infty & \infty & \dots & \infty\\
    |f_1 - x_3| & |f_2 - x_3| & |f_3 - x_3| &  |f_4 - x_3|& |f_5- x_3|& |f_6 - x_3| & \infty&\infty & \dots  & \infty\\
    |f_1 - x_4 & |f_2 - x_4| & |f_3 - x_4| & |f_4 - x_4| &  |f_5 - x_4|& |f_6- x_4|& |f_7 - x_4| & \infty&\dots & \infty\\
    \infty & |f_2 - x_5| & |f_3 - x_5| & |f_4 - x_5| &  |f_5 - x_5|& |f_6- x_5|& |f_7 - x_5| & |f_8 -x_5|&\dots & \infty\\
   &&&& \dots &&&&& \\
\end{bmatrix}$}

\noindent where we have chosen these values because the cost is the absolute difference between the forecast and observations.  
\item In this implementation, the above minimisation gives the adjusted error for each half hour for each household.
\end{itemize}

As for the $4^{\text{th}}-$norm error we can visualise the adjusted error but in order to this we must define the adjusted error for each household as $\hat{E}_4 := \left( \sum_{i=1}^{n} \hat{E}^4\right)^{\frac{1}{4}}$, where $n$ is the length of the forecast. As for the $4^{\text{th}}-$norm error, figure \ref{fig:Adj_err_all} shows the adjusted error for each household for each forecast in black with the flat blue line showing the average. These averages are also explicitly given in table \ref{tab:errs}

\begin{figure}
\centering
\includegraphics[scale=0.5]{Errors_Adj.pdf}
\caption{\label{fig:Adj_err_all} Adjusted error for each forecast.}
\end{figure}


\begin{table}
\centering
\begin{tabular}{|c|c|c|}
\hline
 & $E_4$ & $\hat{E}_e$ \\
 \hline
\textbf{SD} & 3.829 & 3.496 \\
\textbf{LW} & 4.76 & 4.100 \\ 
\textbf{AA} & 4.125 & 3.396 \\ 
\textbf{WA} & 3.866 & 3.542 \\ 
\textbf{BRR} & 3.805 & 3.575 \\
\hline
\end{tabular}
\caption{Mean of $4^{\text{th}}-$norm error and adjusted error.}
\label{tab:errs}
\end{table}


Table \ref{tab:errs} shows the value of the blue lines in each of forecasts using both error metrics. There are many disagreements both qualitatively and quantitatively but let's discuss the similarities first. In both cases, LW forecast is the worst method. This intuitively makes sense since we know LW forecast cannot take into account the week to week variability. While reasonable due to the smoothing effect, it is unexpected that both regression techniques are judged to be less accurate than the SD in either error measure. Note also that regardless of which error measure is used and which of the forecasts is being considered, the error has a range of roughly 3 kWh and 4 kWh. This is quite large considering that the $75^{\text{th}}$ percentile is around the 0.5 kWh.

From table \ref{tab:errs}, it is clear that the choice of metric alters not only the error value but also conclusion of which algorithm is the better. If the $4^{\text{th}}-$norm error metric is used, then the BRR forecast is the best and LW forecast is the worst with the AA forecast being judged the second worst but the same forecast is judged to be the best when the adjusted error measure is used. Based on the qualitative comparison in figures \ref{fig:SDforecast} - \ref{fig:BR_forecast_P1}, our understanding of forecasting electric load is better confirmed by the results from $\hat{E}_4$ and thus for the moment we will use the adjusted error measure to conclude which of the algorithms is better.






 % Forecast and Forecast Validation
\clearpage

\lhead{\emph{Extreme Value Statisitcs}}  % Set the left side page header to "Abbreviations"

\chapter{Extreme Value Analyses} \label{sec:EVT}

As we have noted from the literature review in section \ref{subsec:litrev}, there is no discussion of extremes when it comes to electric load forecasting or about the influence of stresses on the electric grid. Intuitively we can see that the stress may come from a large number of households requiring electricity at the same time, say for breakfast or in the evening but it may also come from a relatively small number of households requiring (unusually) large amounts of electricity, say from a neighbourhood where many households have electric vehicles. The statistics of these extremes is the topic of this chapter.

While classical large sample theory allows the use of the empirical distribution to make some inferences about what happens in the tail behaviour, it fails in cases where the second moment and even the first moment (the variance and mean, respectively) cease to be finite. This is because classical theory is based on the law of large numbers and relies mostly on the normal distribution whose first and second moment are both finite. Additionally, classical theory does not allow the quantification of a probability of an event greater than what has already been observed. This is the strength of extreme value theory (EVT); it offers us techniques that focus on the ``extreme values of a sample, on extremely high quantiles or on small tail probabilities'' \cite[ch.~1]{beirlant}.

\section{Extreme Value Conditions} \label{subsec:EVT}

As a precursor to what follows, let $F$ be a distribution function (d.f.) underlying the population $X$. Assume $X_1,X_2, \ldots, X_n, \ldots$ is a sequence of identically and independently distributed (i.i.d.) random variables with common d.f. $F$. Since there is no essential difference in maximisation and minimisation, we shall consider extreme value theory regarding the maximum of the random sample $(X_1,X_2, \ldots, X_n)$ for a sufficiently large sample size $n$. We shall denote the \textit{sample maximum} by $X_{n,n}$ and define it as $X_{n,n}:= \max(X_1,X_2, \ldots, X_n)$. We can also define the order statistics, $\{X_{1,n} \le X_{2,n} \le ... \le X_{n,n}\}$ and understand the \textit{upper order statistics} to  be $\{X_{n-k,n} \le ... \le X_{n,n}\}$, for suitably small $k$.% to conduct tail inference.% and we shall always be concerned with sample maxima.

The celebrated Fisher and Tippet theorem \citep{ft28}, also known as the Extreme Value theorem, with prominent unifying contributions by \cite{Gnedenko:43} and \cite{deHaan:70}, establishes the Generalised Extreme Value (GEV) distribution as the class of limiting distributions for the linearly normalised partial maxima $\{X_{n,n} \}_{n\geq 1}$. More concretely, if there exist real constants $a_n>0$, $b_n \in \real$ such that
\begin{equation}\label{EVTheo}
	\limit{n} \mathbb{P} \Bigl( \frac{X_{n,n}-b_n}{a_n} \leq x\Bigr)= \limit{n} F^n (a_n x + b_n) = G(x),
\end{equation}
for every continuity point of $G$, then $G(x)= G_{\gamma}(x)$ is given by
\begin{equation}\label{GEVd}
	G_{\gamma}(x)= \exp \{ -(1+ \gamma\, x)^{-1/\gamma}\}, \quad 1+\gamma\,x >0.
\end{equation}
We then say that $F$ is in the (maximum) domain of attraction of $G_\gamma$,  for some extreme value index (EVI), $\gamma \in \real$ [notation: $F \in \mathcal{D}(G_{\gamma}) $]. For $\gamma=0$, the right-hand side is interpreted by continuity as $\exp\bigl\{-e^{-x}\bigr\}$. By taking the logarithm of both sides of the extreme value condition (\ref{EVTheo}) and using Taylor's expansion we have that 


\begin{equation}
-n\log(F(a_n x + b_n )) \approx n(1 - F(a_n x + b_n )) \arrowf{n} (1 + \gamma x )^{-1/\gamma},
\end{equation}

%as $ n \rightarrow \infty$, 
\noindent for those $x$ such that $ 1 + \gamma x > 0 $. The above resonates as follows. We are intersted in extrapolating beyond the available sample, which entails that the mean number of observations above the deterministic threshold $ u_n = a_n x + b_n$ (large enough), given by $ n( 1- F(u_n)) $, must be a very small number. In addition, the threshold $ u_n $ must not be too large so that tail inference can still be carried out, i.e. we assume an \textit{intermediate sequence}, $ k = k_n $ such that $n (1 - F(u_n)) = k$ satisfying
\begin{align} \label{eq:k_cond}
\begin{split}
\lim_{n \rightarrow \infty} k &= \infty \\
\lim_{n \rightarrow \infty} \frac{k}{n} &= 0.
\end{split}
\end{align}

The theory of regular variation \citep{Binghametal:87,deHaan:70, deHF:06} provides necessary and sufficient conditions for $F\in \mc{D}(G_{\gamma})$. Let $U$ be the tail quantile function defined by the generalised inverse of $1/(1-F)$, i.e.


\begin{equation*}
U(t):=   F^{\leftarrow} \bigl( 1-1/t\bigr), \quad \mbox{ for } t >  1.
\end{equation*}


Then, $F\in \mc{D}(G_{\gamma})$ if and only if there exists a positive  measurable function $a(\cdot)$ such that the condition of \emph{extended regular variation}


\begin{equation}\label{ERVU}
	\limit{n}\,\frac{U\left(\frac{n}{k}t\right)-U\left(\frac{n}{k}\right)}{a\left(\frac{n}{k}\right)}= \frac{t^{\gamma}-1}{\gamma},
\end{equation}


holds for all $t>0$ [notation: $U\in ERV_{\gamma}$]. The limit in equation \ref{ERVU} coincides with the $U-$function of the Generalised Pareto (GP) distribution, with d.f. $ 1+ \log G_ \gamma $, which suggests the commonly known improved inference attached to the Peaks Over Threshold (POT) method (described below). In fact, the extreme value condition (\ref{EVTheo}) on the tail quantile function $ u $ is the usual assumption in semi-parametric inference for extreme outcomes. Since we are mainly considering Block Maxima (BM) method, we define the $ k^{\text{th}} $ block maximum to be:

\begin{equation}
M_i = \max _{(i -1) m < j \le i m} X_j.
\end{equation}

\noindent Then we may use the extreme value condition provided in \citet{FdeH:15} which enables us to deal with block length and/or block number as opposed to the number of upper order statistics above a sufficiently high (random) threshold.

\begin{equation} \label{EVTheo2}
\limit{m} \frac{V(mt) - V(m)}{a_m} = \frac{t^\gamma - 1}{\gamma}
\end{equation}


The above states that we are dividing the sample of size n into $k$ block of equal length (time) $m$. For the extreme value theorem to hold within each block, the block length must be sufficiently large, i.e. one $m$ should tend to infinity. Now, let $V$ be the generalised inverse of $ -1/\log F $, i.e. $ V(-1 / \log( 1 - t ) ) = F^{\leftarrow}( 1 - t )$ for $ t > 0 $. Then, $F \in \mc{D} ( G_\gamma )$ if and only if



\noindent for all $ t > 0 $. The extreme value condition in equation \ref{EVTheo2} is the main condition in this report, eventually. Furthermore, by letting $ x = x_m $ such that $ x_m \rightarrow \infty $, as $ m \rightarrow \infty $, in the case where $ \gamma < 0 $, it is possible to devise a class of estimators for the right endpoint of $F$ belonging to some max-domain of attraction, $ \mc{D} (G_\gamma) $.  Namely, condition \ref{EVTheo2} gives rise to the approximate equality $ V(\infty) \approx V(m) - a_m / \gamma $, as $ m \rightarrow \infty $, where $ V(\infty ) := \limit{t} V(t) = F^{\leftarrow}(1) = x^F $, the latter being the right endpoint of $F$ which can be viewed as the ultimate extreme quantile.


%Within the two main approaches in Extreme Value analysis we plan to tackle:
%\begin{enumerate}
%\item\label{BM} BM method: take observations in blocks of equal size and assume that the maximum in each block (time window) follows exactly the Generalised Extreme Value (GEV) distribution defined in equation \ref{GEVd}.
%\item\label{POT} POT method: restrict attention to those observations from the sample that exceed a certain level or threshold, supposedly high, while assuming that these exceedances follow exactly the GP distribution.
%\end{enumerate}

A difficulty in applying EVT is that observations generally do not follow the exact extreme value distribution. The best we can hope for is that they come from a distribution in one of the only possible three domains of attraction. Hence an interesting aspect is to derive large sample properties of the obtained estimators in a % by replacing the word ``exactly'' with ``approximate'' in points \ref{BM} and \ref{POT} above. The latter conveys a 
semi-parametric framework, which also proves to be a  fruitful setting in analysing extreme events. 

%Lastly, what was referred to as the upper limits in the earlier section is known as the right endpoint in this framework and is defined as $x^F := \sup\{x | F(x) < 1\} \le \infty$.

\section{Extreme Value Statistics} \label{subsec:EVres}

Now that we've set up the basic framework, it is useful to consider some graphical tools. A quantile-quantile (QQ) plot answers the typical question: does a particular model plausibly fit the distribution of the random variable at hand? It is relatively simple to construct and can even be done by hand in some cases e.g. for the exponential distribution. For each observation the equivalent quantile of the proposed d.f. $F$ is calculated. Then the ordered data points are plotted against these quantiles to get the QQ plot. Mathematically speaking, the QQ plot is given by the graph \cite[ch.~6]{embrechts}. \newline

\centerline{$\{(X_{k,n},F^{\leftarrow}(\frac{n-k+1}{n+1})), k = \{1,...,n\}\}$}

\noindent where $n$ is number of observations. The linearity in the QQ plots can be used as a goodness of fit assessment between the model and the random variable at hand.

Figure \ref{fig:beta} shows what the QQ plot looks like for the prescribed d.f. $F$ of half hourly electric load data is assumed to be the standard beta distribution (i.e. with parameters $\alpha=1$ and $\beta=1$, which is the same as the uniform distribution). At this point, there is no evidence to suggest that the data are beta (or uniform) distributed  which exposes one disadvantage of the QQ plot; it only tells if a certain distribution is plausible, not which one is correct. Even though, the underlying distribution hasn't been established, it is probable that the data are light tailed, i.e $\gamma < 0$ since both the uniform and beta distributions are light tailed. This will also explain why heavy tailed distributions such as the log-normal and Weibull didn't fit the data well anywhere (not shown). 

\begin{figure}
\centering
\includegraphics[width=0.8\textwidth]{qqplot_beta.png}
\caption{\label{fig:beta} QQ plot.}
\end{figure}

The other graphical tool that is often used in extreme value analysis is a mean excess plot. Mathematically, the mean excess function, $e$,  is defined as

\centerline{$e(t) = \mathbb{E}[X-t | X>t], t \in \mathbb{R}$} 
 
In practice however, the mean excess function can be estimated by its the empirical counterpart, $\hat{e}_n$ \citep[ch.~1]{beirlant}

\centerline{$\hat{e}_n(t) = \frac{\sum\limits_{i=1}^n x_i 1_{(t,\infty)}(x_i)}{\sum\limits_{i=1}^n 1_{(t,\infty)}(x_i)} - t$}

This empirical mean excess function can be plotted against the threshold $t$ or against $k$ which is the number of $x_i$ that exceed the threshold, $X_{n-k,n}$. Figure \ref{fig:beir} shows how the monotonicity of the mean excess function can yield further information about both the distribution and the tail heaviness of the data; an increasing trend for high thresholds is suggestive of heavy tails (such as the log-normal distribution) whereas a decreasing trend of light tails (such as uniform distribution).

\begin{figure}
\centering
\includegraphics[width=0.8\textwidth]{beirfig.png}
\caption{Typical mean excess function for some common distributions. Source: \cite{beirlant}}
\label{fig:beir} 
\end{figure}

Comparing this image with the mean excess plot for the data at hand (fig. \ref{fig:r_me}), there is not any obvious distribution that can be picked out however it is clear that for appropriately high threshold, the mean excess function is decreasing, supporting the earlier deduction that the data may be light tailed. The choice of the appropriate threshold (here considered to be roughly 6 kWh) may seem to be an arbitrary or convenient choice especially when we see increasing behaviour for a threshold of 9 kWh, however choosing such a high threshold also has high variance and thus decreases our confidence in the result. Choosing our threshold to be 6 kWh gives us roughly 370 ``extreme'' observations which is large enough on its own but negligible with respect to the total number of observations available (over of 1 million). Thus when we're dealing with electric load (as opposed to transformed electric load such as returns), we chose $k$ such that $X_{N-k,N} \approx 6$.

\begin{figure}
\centering
\includegraphics[width=0.8\textwidth]{R_meplot.png}
\caption{\label{fig:r_me} Mean excess plot plotted against threshold.}
\end{figure}


Inference for BM has received much attention recently. The relative merits of this approach are discussed in \cite{FdeH:15}, where they lay down several results in line with equation \ref{ERVU}, cementing the path towards the semi-parametric approach to BM estimation.

One of the proposed maximum Lq-likelihood estimator (MLqE) of $\theta$ (real or vector-valued parameter) is defined as
\begin{equation*}
	\hat{\theta}= \argmax{\theta \in \Theta} \, \sumab{i=1}{n} L_q\bigl( f(X_i; \, \theta)\bigr), \quad q>0,
\end{equation*}
with $L_q(u)= \log u$ if $q=1$ and $L_q(u)= (u^{1-q}- 1)/(1-q)$, otherwise (cf. Definition 2.2 in Ferrari and Yang, 2010). The function $L_q$ is quite similar to the Box-Cox transformation in statistics. The parameter $q$ gauges the degree of distortion in the underlying density. If $q=1$, then the estimation procedure reads as the ordinary maximum likelihood (ML) method.

The rationale to the maximum product spacing (MSP) estimator can be found in \cite{ChengAmin:79} and \cite{Ranneby:84}. The MSP estimator is given by

\begin{equation*}
\hat{\theta} = \argmax{\theta \in \Theta} \sumab{i=1}{n} \log \left( F(X_{j,n};\theta) - F(X_{j-1,n};\theta) \right),
\end{equation*}

\noindent with $F(X_{0,n}) = 0$ and $F_(X_{n+1,n}) = 1$. %where $F(x;\theta)$ and $G(x)$ are distribution functions of $f(x;\theta)$ and $g(x)$ respectively. $g(x)$ is the true but unknown density function of the random variables $X_1, X_2, ... $ where as $f(x;\theta)$ is the density function of the model assigned to the same. Both the MSP and ML estimators are related by the fact that they are approximating the same information, though in separate ways \citep{Ranneby:84}. The information that is being approximated is known as the Kullback-Leibler information, $I(g,f) = \int g(x)\log( g(x)/f(x) ) d\lambda(x)$, where $\lambda$ is some measure.
The MSP estimator looks at spacings between subsequent observations whereas the ML estimators looks at the observations.

Now we are going to apply both of these estimation methods (MLqE and MSP) to the available sample (7 weeks) of weekly maxima, which are assumed to follow a GEV by equation \ref{GEVd}. We choose weekly maxima so that assumption of independence may hold. Note that in chapter two (fig. \ref{fig:acf_day}) we explored the dependent structure of averages and exploited tis in chapter 3 for each of the forecasting methods. However, in the most extreme observations, this structural dependence becomes week and the extreme value conditions still hold. Since we have few extreme (max-)data we need to replicate. This will be done by drawing on the 503 customers' meter readings, thus we will have $7 \times 503$ extreme data points.

Figure \ref{fig:gammaEst} displays the estimates for the EVI, $\gamma$ (or shape parameter), using both Lq-likelihood and MSP estimation procedures upon the BM method. We are not so much concerned with the estimate of $\gamma$ but the sign of the estimate of $\gamma$, which we see is negative. Thus, we have rigorously confirmed our earlier deduction that $\gamma <0$, though this rigor holds only for weekly maxima and not necessarily for other block maxima. Intuitively, data being light-tailed means that the sum of the tail probabilities is small which in turn means that the right endpoint is finite. This brings us neatly to the right endpoint estimation. The estimation requires an input for $\gamma$, thus the values shown in figure \ref{fig:gammaEst} are used in the estimation of the right endpoint (fig. \ref{fig:EndPointEst}). From these two implementations if the MLqE, the red one being the more conservative estimate of the two, we get that the right endpoint for weekly maxima is between 12-13.5 kWh. This makes intuitive and contextual sense; most DNO impose an upper limit for households by contract and exceeding this limit may cause a fuse to blow or a blackout to occur thus giving a physical limit to what an electric grid may be able support.  Reassuringly, the estimate of 13.5 kWh is actually quite close to the 15 kWh upper limit imposed by DNO on residential customers.

\begin{figure}
\begin{center}
\includegraphics[scale=0.7]{GammaEstimates.pdf}
\caption{Estimates for the shape parameter based on the weekly maxima.} \label{fig:gammaEst}
\end{center}
\end{figure}

\begin{figure}
\begin{center}
\includegraphics[scale=0.7]{EndpointEst.pdf}
\caption{Estimates for the right endpoint based on the weekly maxima.} \label{fig:EndPointEst}
\end{center}
\end{figure}

Figure \ref{fig:POTEst} displays the semi-parametric estimation results while adopting the Moment (M) and Mixed Moment (MM) estimators in connection with the POT method. Notice that, for the sake of consistency, we have chosen the threshold to be 6 kWh. Even though the POT method is not one that we are currently pursuing, it adds to the evidence that the data are indeed light tailed. Except for $k$ very small, where variance is too large and data too sparse to conduct any inference with confidence, we have that $\gamma < 0$.

\begin{figure}
\begin{center}
\includegraphics[scale=0.7]{EVIestimation.pdf}
\caption{Extreme value index estimation using the $k$th larger order statistics.} \label{fig:POTEst}
\end{center}
\end{figure}

\section{Heteroscedasticity of Extremes} \label{subsec:sced}

The discussion in section \ref{subsec:EVT} describes ``classical'' extreme value theory, the foundation of which requires the data to be i.i.d. However, this may not always be the case in reality. \cite{einmahl16} described the behaviour of the extremes which do not subscribe to the notion of identically distributed, namely heteroscedastic extremes. The mathematical set up is as follows. Suppose we have independent observations taken at $n$ time points, $X_1^{(n)} , ... , X_n^{(n)}$, which have different distribution functions $F_{n,1}, ... , F_{n,n}$. Each of the $n$ observations however have the same right endpoint, denoted by $x^F \in (-\infty, \infty]$. Furthermore, if the distribution function $F$ also shares this right endpoint, then the \textit{scedasis function}, denoted by $c$ and defined (eq. \ref{eq:scedasisfn}) on $[0,1]$, is indicative of the frequency of extremes.
% The condition presented in equation \ref{eq:scedasis_cond} ensures that the definition presented in equation \ref{eq:scedasisfn} is unique $\forall n \in \mathbb{N}$ and $\forall 1 \le i \le n$.

\begin{equation} \label{eq:scedasisfn}
\lim_{x \rightarrow x^F} \frac{1-F_{n,i}(x)}{1 - F(x)} = c\left(\frac{i}{n}\right).
\end{equation}

\noindent This scedasis function becomes uniquely defined  $\forall n \in \mathbb{N}, 1 \le i \le n$ when the density condition (eq. \ref{eq:scedasis_cond}) is imposed.


\begin{equation} \label{eq:scedasis_cond}
\int_0^1 c(s)d(s) = 1.
\end{equation}

The assumption required for this model is on $F  \in \mathcal{D}(G_{\gamma})$ belonging to the maximum domain of attraction of the GEV distribution and while it can be shown that $F_{n,i}  \in \mathcal{D}(G_{\gamma})$, this does not need to be assumed \textit{a priori}. The theory presented in \cite{einmahl16} is developed for positive EVI. Since we know our data is light tailed, any results using an estimate of $\gamma$ cannot be applied directly. Estimators for $\gamma <0$ exist \citep{Ferreira17} but these need to be adapted before. This adaptation will be pursued as future work. %, i.e. $\gamma > 0 $ which means that the Hill estimator (eq. \ref{eq:hill_est}) was used to estimate the EVI.
%
%\begin{equation} \label{eq:hill_est}
%\hat{\gamma}_H := \frac{1}{k} \sum_{j=1}^k log(X_{n,n-j+1}) - log(X_{n,n-k}),
%\end{equation}
%where $k=k(n)$ is a suitable intermediate sequence. The Hill estimator is not suitable for $ \gamma \in \mathbb{R}$ however other appropriate exist. For example, \citet{Ferreira17} developed an estimator for $\gamma < 0 $ in the spatial-temporal setting, which can be adapted for just the temporal setting as in our context. 

Despite the fact that the work presented in \cite{einmahl16} is for $\gamma >0$, we can still apply some of the results and tests for electric demand. Let us first look at some of the results. %\cite{einmahl16} splits their study into three main sections that are of significance to this report. In the first of these section, some properties of $c$ are deduced and $\gamma$ is estimated. 
The first result that is relevant to us is the estimator for the scedasis function (eq. \ref{eq:c_hat_est}).

\begin{equation} \label{eq:c_hat_est}
\hat{c}(s) = \frac{1}{kh} \sum_{i=1}^n \mathbb{1}_{\{X_i^{(n)} > X_{n,n-k}\}}G \left(\frac{s-\frac{i}{n}}{h} \right),
\end{equation}

\noindent where $G$ is a continuous, symmetric kernel on $[-1,1]$ s.t. $\int_{-1}^{1} G(s)ds = 1$ and $G(s) = 0 \quad \forall s \notin [-1,1]$ , $h := h_n$ is known as the bandwidth s.t. $h \rightarrow 0$ and $kh \rightarrow \infty$ as $n \rightarrow \infty$. the biweight kernel, $G(s) = \frac{15(1-x^2)^2}{16}, x \in [-1,1]$, was used throughout the study. The kernel has value 0 for $s \notin [-1,1]$.

The second result that is of importance is the two test statistics, $T_1$ and $T_2$ (eq. \ref{eq:test_stat12}).  

\begin{align} \label{eq:test_stat12}
\begin{split}
T_1 : =& \sup_{0 \le s \le 1} |\hat{C}(s) - C_0(s)|,\\
T_2 : =& \int_0^1 \{\hat{C}(s) - C_0(s)\}^2 dC_0(s),\\
\end{split}
\end{align}

\noindent where $C_0(s) = \int_0^s c_0(u)du$ for $c_0$ given. We start by implementing the scedasis estimator, $\hat{c}$, (eq. \ref{eq:c_hat_est}) and then test for the trend. These are used to reject the presence of homoscedastic extremes ($H_0: c \equiv 1$) under the assumption that $\gamma$ is constant. \cite{einmahl16} provided statistics to test for the invariance of $\gamma$ but these utilise the Hill estimator which is inappropriate for the light tailed data at hand. The adaptation of these test for negative EVI will be pursued in the future. We will assume for the time being that $\gamma$ is constant and test for the trend in the scedasis function.


%\begin{align}  \label{eq:test_stat34}
%\begin{split}
%T_3 := & \quad \sup_{0 \le s_1 < s_2 \le 1, \hat{C}(s_2) - \hat{C}(s_1)} \left|\frac{\hat{\gamma}_{(s_1,s_2]}}{\hat{\gamma}_H} -1\right|, \\
%T_4 := & \quad \frac{1}{m} \sum_{j=1}^m \left(\frac{\hat{\gamma}_{(l_{j-1},l_j]}}{\hat{\gamma}_H}-1\right)^2,
%\end{split}
%\end{align}
%
%\noindent where  $\hat{\gamma}_{(s_1,s_2]}$ is the Hill estimator based on $X_{[ns_1]+1}^{(n)}, ... , X_{[ns_2]}^{(n)}$,  $l_j : = \sup\{s: \hat{C}(s) \ge j/m\}$, where $m$ is the number of block the full sample has been divided into, and the estimator for $C = \int_0^s c(u) du $ is given by $\hat{C}(s) : = \frac{1}{k} \sum_{i=1}^{[ns]} \mathbb{1}_{\{X_i^{(n)} > X_{n,n-k}\}}$. Note that both test statistics $T_3$ and $T_4$ use the Hill estimator for $\gamma$, which is unsuitable for our use. More work needs to be done in order to adapt these tests for negative EVI, which will be pursued in the future.

%Test statistics $T_3$ and $T_4$ use the Hill estimator and are not applicable to our data, thus we will not test for the trend in $\gamma$. Instead 

 \citet{einmahl16} applied these results to daily loss returns from the Standard and Poor's 500 Index (SP500) and found that $\gamma$ was constant for certain years and confirmed the presence of heteroscedastic extremes i.e. $c \neq 1$. We now adapt this work. SP500 data is one time series whereas we have 503 time series, one for each household. Thus, we proceed as follows.
\begin{enumerate}
\item Data is at half hourly resolution which means that for each of 503 households, there are 2352 measurements.
\item At each half hour, the maximum over household is taken. In this way we get a single time series with length 2352.
\item The choice of $k$ is often heuristic and application dependent. The correct choice of $k$ has not been yet been evaluated. Since we chose the global threshold to be roughly 6 kWh, we choose a $k$ for this data so that $X_{N-k,N} \approx 6$ also which gives us $k=353$ however we end up using $k=400$ as it is common practice to round up.
\item The scedasis is then estimated using the biweight kernel and a bandwidth of 0.1. Similar to $k$, choosing the bandwidth, $h$, is not trivial and is application specific. For the purpose of this report we have used the same $h$ as in \citet{einmahl16}.
\item The resulting image, evaluated using equation \ref{eq:c_hat_est}, is given in figure \ref{fig:mysced_hh_max}.
\end{enumerate}

%The second section of the study also ran simulations using various data generating process so as to validate the model for various kinds of distributions. In these simulations, where the data length was $n=5000$, $k$ was chosen to be 400 and the other parameters were $h = 0.1$ and $G(s) = \frac{15(1-x^2)^2}{16}, x \in [-1,1]$. This kernel is known as the biweight kernel, which is taken to be zero for $ s \notin [-1,1]$.

%The final section of \cite{einmahl16} that is of importance to us is the application of the above estimators and tests to financial data. To be specific the authors started with daily loss returns of the SP500 index from 1988 to 2012. This sample had 6302 observations with 2926 days of losses and tests were conducted using $k=160$. This data included the financial crisis that erupted in 2008 which may have to lead to the lack of significant results. Thus the authors used a subsample, from 1988 to 2007 which included 5043 observations and 2348 days with losses. In this case, $k$ was chosen to be 130 and it was shown that the null hypothesis ($H_0 = \gamma$ constant) could not be rejected (using tests $T_3$ and $T_4$) whereas the frequency of extremes being invariant was rejected (using tests $T_1$ and $T_2$).  It was also noted that the scedasis function, generated using the subsample, showed a sharp increase at the end of 2007 even before the financial crisis occurred. This final analysis is relevant to our future work since much like electric load data financial data are light tailed. However, financial returns (such as loss returns as used here) are heavy tailed which may also be the case for transformations of electric load data. Moreover, this data set is also a time series much like ours and we too will consider properties and heteroscedasticity of returns. Much like what the authors have done here, we too will eventually consider quantifying changes, if any, in the frequency of extremes. Understanding how the behaviour of extremes changes, specifically in their frequency, allows Distribution Network Operators (DNO) to infer new installations of appliances in homes or of photovoltaic (PV) cells and purchases of electric vehicles thereby supporting the personalisation of electricity plans and contracts for better customer care and service. Features such as the sharp increase in the scedasis function may also inform DNO of impending risk of power failure.

\begin{figure}
\centering
\includegraphics[scale=0.55]{hh_max_sced(1).pdf}
\caption{\label{fig:mysced_hh_max} Proportion scedasis for half hourly max over 503 households for a period of 7 weeks using $k=400$ and $h=0.1$ where $n=2352$.}
\end{figure}

%Both estimations of the scedasis function (\ref{fig:mysced_hh_max}) are reasonably consistent with each other except perhaps the middle peaks; the biweight kernel yields 5 peaks whereas the Epanechnikov kernel yields 4. For the most part peaks occur roughly at the same time and as do the troughs. 

The scedasis of the half hourly maxima (fig. \ref{fig:mysced_hh_max}) has 5 main peaks. It is valuable to see if there are any patterns regarding the timing of these peaks. The first peak occurs roughly around day numbered 597 which is a Friday. Recall from figure \ref{fig:sums} that collectively Friday was the day of the week which was had the smallest electric load however this result is not entirely discouraging; it demonstrates that extreme values may behave differently to sums and averages which we know to be true and is the reason that a theory for extremes exists. The other peaks occur roughly on weekends and thus better aligns with what we saw in figure \ref{fig:sums}: the second peaks occurs around day numbered 605 (Saturday), the third occurs between days numbered 617 (Thursday) and 621 (Monday), the fourth between 625 (Friday) and 629 (Tuesday) and the last peak between 633 (Saturday) and 637 (Wednesday). The last two in these cases are admitted closer to their lower ranges than the higher ranges which confirms that it is not just overall high usage that occurs on weekends but individual peaks are also more likely to occur on weekdays. Also it is notable that although we observe peaks on weekends, we don't have a peak at every weekend. Having said that, it is encouraging that there are more peaks later in the time series as we did see an increase in usage towards the end of the weekly total demand.

Let's come back for a moment to the concept of scedasticity. Scedasis is not necessarily a measure of where the most extreme event is likely to occur or has occurred rather an indication of where extreme values are more frequent and it's a somewhat relative scale. This means that we may have two households where household 1 has a larger scedasis at a certain time than household 2 at the same time but the load at the time may be larger for household 2. This is because household 1 is simply using more energy than is ``normal'' at that point in time where as the load from household 2 is relatively (to itself) not as extreme.% (we'll see this later). 
We can look at a moving average of the half hourly maxima and for large windows, say 100 or 200 half hours, we notice a somewhat similar pattern to figure \ref{fig:mysced_hh_max}. \todo{include image later} All the peaks occur at roughly the same time and roughly in similar magnitudes. While this may a rudimentary comparison it helps to conceptualise relative risk that the scedasis function describes.

For the same data set, the integrated scedasis function can be seen in figure \ref{fig:myintsced_hh_max}. The image in itself is not what is of consequence here but the use of these values to apply tests $T_1$ and $T_2$ given in equation \ref{eq:test_stat12}.

\begin{figure}
\centering
\includegraphics[scale=0.6]{hh_max_int_sced.pdf}
\caption{\label{fig:myintsced_hh_max} $\hat{C}$ for half hourly max over 503 households for a period of 7 weeks with $k=400$.}
\end{figure}

Now that we have established an estimate for $\hat{C}$ and $\hat{c}$, we can apply the Kolmogorov-Smirnov-type test statistic ($T_1$) and Cramer-Von-Mises-type test statistic ($T_2$). We are testing to see if there is a trend in the scedasis function thus the null hypothesis is $H_0: c_0 =0$. This gives that $C_0(s) = s$. Then we get that $T_1 = 0.675$ and $T_2 = 0.153$ which when normalised appropriately (cf Corollary 1 in \citet{einmahl16}) we get 13.5 and 61.21 respectively. We can look up the corresponding critical values in \citet{tables1,tables2} and we get p-values which are virtually zero. Thus we reject the null hypothesis and establish the presence of heteroscedastic extremes i.e. there is a trend in $c$. This is important because we now know that our earlier assumption that extremes come from the same distribution is not correct. On the other hand this result provides good motivation to further develop parametric estimators in the setting where the data are not identically distributed.

We can continue and consider analogies for the daily losses considered in \cite{einmahl16}, e.g. the positive differences between some aggregate statistic for day $d$ and the same aggregate statistic for day $d-1$ for each household which can be aggregated again in some way to get one time-series. However the data set we are currently considering is only 7 weeks long i.e. 49 days so we are limited but it's a valuable start nonetheless.

Before scedasis function is estimated for various positive differences, it is useful to review what aggregating statistics have been used. The following were thought to be relevant either from an electric load perspective or an extremes perspective: \begin{enumerate*}[label=\roman*)] \item maximum, \item mean, and \item sum. \end{enumerate*} Using these, 4 different positive differences data sets were created in the following way:
\begin{enumerate}
\item The data, which contains measurements at half hourly resolution for 503 customers, was grouped by day.
\item For each house the mean, maximum or total electric load was recorded for each day.
\item Then to get one time series as before, a maximum or a sum over all households were recorded.
\item Thus 4 positive differences were used. The naming system in figures \ref{fig:pos_diff} and \ref{fig:pos_diff_sced} represent which kind of aggregation has been applied. For example ``Total of Daily max'' indicates the  daily max is summed over all households whereas ``max of daily max'' indicates a maximum over all households is acquired from daily maxima.
\item What these look like in terms of measurements is shown in figure \ref{fig:pos_diff}.
\item The corresponding estimated scedasis function is given in figure \ref{fig:pos_diff_sced}.
\end{enumerate}

\begin{figure}
\centering
\includegraphics[scale=0.85]{pos_diffs.pdf}
\caption{\label{fig:pos_diff} Profiles of various positive differences.}
\end{figure}

\begin{figure}
\centering
\includegraphics[scale=0.5]{pos_diff_sced(1).pdf}
\caption{\label{fig:pos_diff_sced} Estimated scedasis functions, $\hat{c}$ for various positive differences with $k=20$ (rough scaling), $h=0.1$ and using the biweight kernel at daily resolution.}
\end{figure}

As we did above, we can look at where the peaks occur for positive differences. There are seven peaks for most of positive differences. It's only the scedasis of the ``Max of Daily Max'' that has 5 peaks and maybe another point of inflexion. For those that have 7 peaks, the $4^{\text{th}}$ and $5^{\text{th}}$ peaks are very close and could arguably be part of the same. First let's look at when each of the peaks occurs (table \ref{tab:pos_diff_sced}). Table \ref{tab:pos_diff_scedd} related the day numbering to the day of the week. Clearly some of these occur in the middle of the week but most occur over weekends and there is not a clear pattern. Given that cumulatively Saturdays and Sundays tend to be most intensive week and Thursdays and Fridays have the least usage (fig. \ref{fig:sums}) thus we would expect a lot of positive differences to peak on Saturday but this would only be strictly true if we were considering general behaviour, not necessarily peak behaviour. When we consider daily totals, a lot of the peaks do occur on the Saturdays as expected or at least on the weekend but some occur on weekdays too and indiscriminately so. This is because extreme behaviour need not follow average/general behaviour and also individual peak behaviour need not follow cumulative peak behaviour.  For example, ``Max of Daily Max'' follows individual extreme behaviour as does ``Max of Daily Total''. In contrast the ``Total of Daily Total'' picks out collective peak behaviour along with ``Total of Daily Max''. Most of the peaks of ``Max of Daily Max'' occur directly after the corresponding peaks in ``Total of Daily Total'' showing how the cumulative peak can differ from the individual peak. Qualitatively speaking it also shows that the peak individual load, in this case transformed load, carries on after the peak cumulative load but does not necessarily contribute to it. This makes sense; it's quite common that individual households use a lot of energy, say outside of peak hours and unless enough households are also doing the same, it does not put strain on the network.


\begin{table}
\centering
\begin{tabular}{|c|c|c|c|c|}
\hline
%Peak & Total of Daily Total & Max of Daily Max & Max of Daily Total & Total of Daily Total \\
 & & & & \\
 \textbf{Peak} & \textbf{\shortstack{Total of \\Daily Total}} & \textbf{\shortstack{Max of \\Daily Max}} & \textbf{\shortstack{Max of \\Daily Total}} & \textbf{\shortstack{Total of \\Daily Total}} \\
  & & & & \\
 \hline
1 & 595-596 & 596-597 & 594-595 & 595-596 \\
2 & 604-605 & 606 & 603-604 & 605-606 \\
3 & 611 & 614 & 612-614 & 611-612 \\ 
4 & 620 & & 617-619 & 622 \\
5 & 625 & & 624-625 & 624 \\
6 & 633-635 & 630 & 633-635 & 633-635 \\
7 & 639 & 639-641 & 640 & 638 - 640 \\
\hline
\end{tabular}
\caption{Peaks in the estimated scedasis functions of various positive difference.}
\label{tab:pos_diff_sced}
\end{table}

\begin{table}
\centering
\begin{tabular}{|c|c|c|c|c|}
\hline
 & & & & \\
 \textbf{Peak} & \textbf{\shortstack{Total of \\Daily Total}} & \textbf{\shortstack{Max of \\Daily Max}} & \textbf{\shortstack{Max of \\Daily Total}} & \textbf{\shortstack{Total of \\Daily Total}} \\
  & & & & \\
 \hline
1 & Wednesday- Thursday & Thursday - Friday & Tuesday-Wednesday & Wednesday-Thursday \\
2 & Friday-Saturday & Sunday & Thursday-Friday & Saturday - Sunday \\
3 & Friday & Monday & Saturday-Monday & Friday-Saturday \\ 
4 & Friday-Sunday & & Thursday-Saturday & Tuesday \\
5 & Friday & & Thursday-Friday & Thursday \\
6 & Friday-Sunday & Wednesday & Friday-Sunday & Friday-Sunday \\
7 & Friday & Friday-Sunday & Saturday & Thursday-Saturday \\
\hline
\end{tabular}
\caption{Day of the week of the peaks of estimated scedasis functions for various positive difference.}
\label{tab:pos_diff_scedd}
\end{table}

Thus far we have used a density scedasis but this quite wasteful of the data. We can use more of the data by considering the proportion of houses which exceed a global threshold. The advantage of this approach is that which can relate more easily to probability than the density scedasis estimator. The density scedasis does not directly translate to a probability since locally it may be more than one (it still tells us how frequent the extremes are at a time $j/n$) however using the frequency of exceedances and proportion scedasis, it is more straightforward to interpret as a probability.

Thus, a crude estimation of the number of households which exceed a global threshold at time $j/n$ can be calculated and interpreted as a scedasis estimator based on proportion. Suppose we have $m = 503 $ household which are measured a $n = 2352$ time points thus we have $N= n \times m$ observations. Recall the scedasis assumption reformulated as shown below holds uniformly in $i$ and $n$, $\forall j$ satisfying $\int_0^1 c(t)dt = 1$

\begin{equation*}
\frac{\mathbb{P}(X_i(j)>x)}{1-F(x)} \quad  \displaystyle{ \mathop{\rightarrow}^{x \uparrow xF}} \quad c\left(\frac{j}{n}\right)
\end{equation*}
\noindent Thus we replace the density scedasis estimator, $\hat{c}$, by the \textit{proportion scedasis}, denoted by $\hat{p}$, which is the proportion of households that exceed a global threshold of $X_{N-k,N}$ at time $j/n \equiv s_j$. Thus

\begin{align} \label{eq:sced_prop}
\begin{split}
\hat{p}(s_j) =  \frac{k_j}{k} =& \quad \frac{\# \text{exceedances at } \frac{j}{n}}{k} \\
= & \quad \frac{\sum_{i=1}^m \mathbb{1}_{\{X_i(s_j) > X_{N-k,N}\}}}{k}\\
\end{split}
\end{align}

Implementing this in R for the 7 weeks of data at the half hourly time resolution (i.e. $N = 1185498$), the frequency scedasis obtained is presented in figure \ref{fig:hh_sced_prop}. Here the $k$ was chosen heuristically to use about 20\% of the data. Note that a line plot is no longer presented to reflect the discrete nature of $\hat{p}$.

\begin{figure}
\centering
\includegraphics[scale=0.5]{hh_sced_prop_full.pdf}
\caption{\label{fig:hh_sced_prop} Scedasis using the frequency of exeedances (eq. \ref{eq:sced_prop}) for 7 weeks with $k=24000$.}
\end{figure}

\section{Heteroscedastic Extremes in Forecasts} \label{subsec:sced_forecast}

We have yet to connect forecasts and forecast errors with scedasticity so let's do this now. Picking the AA forecast, the SD forecast and the WA forecast, we can estimate the scedasis function of errors (absolute difference between forecast and observation). As before week 22 is forecasted for 503 households thus we have $336 \times 503 = 169008$ measurements. For each error a mean excess plot (figs. \ref{fig:AA_err_me}, \ref{fig:SD_err_me}, \ref{fig:LR_err_me}) has been created so we may choose an appropriate value of $k$. Looking at these images, an error of 5kWh was chosen as appropriately extreme error in each case. From this $k$ was deduced to be 58, 46 and 48 for the AA, SD and LR errors, respectively. Thus our choice of $k$, by rounding up, is 100. %Intuitively we know this is a very high threshold to have since the median observation of electric load is around the 5 kWh however using 4kWh means then that $k=200$ which would no longer satisfy the second condition of equation \ref{eq:k_cond}. 

\begin{figure}
\centering
\includegraphics[scale=0.5]{AA_err_meplot.pdf}
\caption{\label{fig:AA_err_me} Mean excess plot of the absolute difference between the AA forecast and observation for week 22.}
\end{figure}

\begin{figure}
\centering
\includegraphics[scale=0.5]{SD_err_meplot.pdf}
\caption{\label{fig:SD_err_me} Mean excess plot of the absolute difference between the SD forecast and observation for week 22.}
\end{figure}

\begin{figure}
\centering
\includegraphics[scale=0.5]{LR_err_meplot.pdf}
\caption{\label{fig:LR_err_me} Mean excess plot of the absolute difference between the WA forecast and observation for week 22.}
\end{figure}

Let's then look at the scedasis functions of these errors (figs. \ref{fig:AA_err_sced}- \ref{fig:LR_err_sced}). All three of these estimated scedasis functions are interesting because they all contain 7 peaks (though admittedly some are not as strong as the others). Recall that there are seven days worth of data at half hourly resolution. The labels of the x-axis have been placed so that it indicates the noon of that day. Thus each of these images is telling us that we're most unsure of what is happening around from noon to night time or that errors are particularly high for these times of the day. Moreover, the last two peaks are particularly pronounced meaning that it there relatively large errors for Saturday. All of this is particularly worrying because most of the high cumulative usage is exactly in these hours (figs. \ref{fig:sums} and \ref{fig:days}). However we can take comfort from knowing that this is not because of any one forecasting method. Perhaps because of the varied usage and peak hours of usage, these periods of the day will always be susceptible to higher errors that other periods of the day. Not only, perhaps this may further establish the benefits of clustering usage as it may help us to better predict usage at precisely these periods of the day.

\begin{figure}
\centering
\includegraphics[scale=0.5]{AA_err_sced.pdf}
\caption{\label{fig:AA_err_sced} Estimated scedasis function of AA forecast errors.}
\end{figure}

\begin{figure}
\centering
\includegraphics[scale=0.5]{SD_err_sced.pdf}
\caption{\label{fig:SD_err_sced} Estimated scedasis function of SD forecast errors.}
\end{figure}

\begin{figure}
\centering
\includegraphics[scale=0.5]{LR_err_sced.pdf}
\caption{\label{fig:LR_err_sced} Estimated scedasis function of WA forecast errors.}
\end{figure}

We have a covered a lot of ground in this chapter. We confirmed the light tailed nature of (weekly-) maxima which we had suspected due to the exploratory analysis in section \ref{subsec:EVres}. This means that we have a finite right endpoint, i.e. a value that bounds the most extreme data. This makes sense from a contextual point of view too since DNOs tend to put a contractual obligations on customers to use less than a certain amount of electricity at any given time and while this may be exceeded without the occurrence of a power outage there exists a physical limit to how much a household can use before the fuse blows. As we discussed before, if longer data were available for each household the same analyses could be done to find a right endpoint so that a tailored contract may be provided to each customer and thus reducing both the cost of electricity for the customer and the amount of energy DNOs need to supply at any given time. The pitfall of these results is that we assumed our data to be i.i.d. which we showed in the last section was not the case. In doing this part we established that there is a trend in extremes meaning that the frequency of extremes changes in time. With realistic modelling of physical, demographic and meteorological variables, we can study these better and use it to forecast load more efficiently and in turn also effectively incentivise customers to reduce demand. Lastly we looked at how the errors were connected to the scedasis function. We observed increased frequency of large errors coinciding with when most usage was high regardless of the forecast used which presents both a challenge and an opportunity to make real  world improvements in electric load forecasts.



 % Extreme Value Theory
\clearpage

%\lhead{\emph{Extreme Value Statisitcs}}  % Set the left side page header to "Abbreviations"

\chapter{Conclusion}

We started with two objectives. The first objective was to set up a benchmark forecast and benchmark error measure. We looked at five different forecasts: the SD, LW, AA, WA and the BRR forecast.  We validated each of these forecasts using 2 error measures, the $4^{\text{th}}-$ norm error and the adjusted $4^{\text{th}}-$ norm error. In this report, we had started with the SD forecast which we found to be, on average,  the best of the five forecasts or the second best and due to its popularity in the literature, this was thought to be good candidate for the benchmark forecast. Another good candidate is the AA forecast since at least qualitatively, but also quantitatively if the adjusted $4^{\text{th}}-$ norm error is used, the best forecast; it produces a forecast where the peaks are better represented than the others. Forecasts such as the WA and the BRR suffered due to the over-smoothing of the profiles and the LW forecast does not allow for week-to-week variability. Due to the widespread usage and transparency, both the SD forecast and the $4^{\text{th}}-$ norm error are established as the benchmark. It should thus be noted that using other forecasts (AA) and error measures (adjusted $4^{\text{th}}-$ norm), we established better forecasts and better methods of judging the goodness of said forecasts during this project.

The second objective was to apply results from extreme value theory to electric load data. We used both the BM method and the POT method to estimate the EVI and the BM method to estimate the right endpoint for weekly maxima. From this, we gained that weekly maxima are light tailed and have a right endpoint between 12 and 13.5 kWh. If longer historic data is available, DNOs can do the same analysis for each household to estimate the right endpoint thereby generating personalised electricity contracts and limits. Similarly, business can evaluate if they have enough headroom in their current plans and whether they should invest in electricity storing devices. We then went further and looked the frequency of extremes by relaxing the assumption of identically distributed. From this we established the presence of heteroscedastic extremes, meaning that half hourly maxima are not identically distributed. Crucially we looked at the scedasis function. Again DNOs can look at the scedasis function for each household and identify any major changes to usage such as purchase of electric vehicles, PV cells, etc which is significant when it comes to demand response and customer service. Finally we started to connect forecasting with EVT by considering the scedasis function of forecasting errors. From this we learnt that high usage was inherently linked to high forecasting errors and subsequently concluded that perhaps introducing clustering may be one way to tackle this problem.

While we have set a benchmark and made improvements on it, it has come at a price. Both the AA forecast and the adjusted $4^{\text{th}}-$ norm error measure in their current implementation are time consuming. Given the strength of the forecast and the error measure, it is viable and beneficial to find an alternative (to the Hungarian algorithm) to solve the minimisation problem for the local permutations in time. Not only, this we have not yet introduced impacts of temperature, rainfall or modelled social and demographic factors. This is a valuable venture as we want the forecast to be realistic and not just reflect the typical day. By combining existing forecasts with EVT may help us to generate forecasts that represent the peaks in load accurately. Knowing when peaks are going to occur and the magnitude of those peaks will help DNOs be prepared for them. Without knowing when peaks in demand will occur and how large those peaks will be means that DNOs will always have to ensure that the network can supply as much during non-peak hours of the day as during the peak hours of the day. However with analyses such as ours, DNOs can do better demand response and thus reduce both consumption and subsequently generation thereby reducing overall carbon emissions. %tpeak load is can use this information to effectively incentivise customers to spread their usage over longer periods of the day. It is important to realise that DNOs must be able to provide the maximum demand at any given time to avoid the possibility of a power outage which means that even during non-peak

While the application of EVT to electric load data is novel, everything we have presented so far is non-parametric. Since we are motivated by industrial applications and want to reflect reality as closely as possible, it is important and necessary to introduce the parametric framework in the adaptation of EVT to electric load. This is the natural progress for this work. Combining this with the modelling we spoke of before, it will be possible to do climate change experiments and see how the EVI, right endpoint, etc. may change in the future. Results such as this will help DNO and policy makers to conduct timely and cost effective upgrade and maintenance to the electric. Ultimately we want our analyses to help make informed decisions and we want to have confidence in these analyses especially when we are uncertain about the future. It is also important that the decisions support local and national policy to combat anthropogenic climate change by collectively reducing carbon emissions. % Conclusion
\clearpage

%% ----------------------------------------------------------------
% Now begin the Appendices, including them as separate files

%\addtocontents{toc}{\vspace{2em}} % Add a gap in the Contents, for aesthetics
%
%\appendix % Cue to tell LaTeX that the following 'chapters' are Appendices
%
%\chapter{An Appendix}

Lorem ipsum dolor sit amet, consectetur adipiscing elit. Vivamus at pulvinar nisi. Phasellus hendrerit, diam placerat interdum iaculis, mauris justo cursus risus, in viverra purus eros at ligula. Ut metus justo, consequat a tristique posuere, laoreet nec nibh. Etiam et scelerisque mauris. Phasellus vel massa magna. Ut non neque id tortor pharetra bibendum vitae sit amet nisi. Duis nec quam quam, sed euismod justo. Pellentesque eu tellus vitae ante tempus malesuada. Nunc accumsan, quam in congue consequat, lectus lectus dapibus erat, id aliquet urna neque at massa. Nulla facilisi. Morbi ullamcorper eleifend posuere. Donec libero leo, faucibus nec bibendum at, mattis et urna. Proin consectetur, nunc ut imperdiet lobortis, magna neque tincidunt lectus, id iaculis nisi justo id nibh. Pellentesque vel sem in erat vulputate faucibus molestie ut lorem.

Quisque tristique urna in lorem laoreet at laoreet quam congue. Donec dolor turpis, blandit non imperdiet aliquet, blandit et felis. In lorem nisi, pretium sit amet vestibulum sed, tempus et sem. Proin non ante turpis. Nulla imperdiet fringilla convallis. Vivamus vel bibendum nisl. Pellentesque justo lectus, molestie vel luctus sed, lobortis in libero. Nulla facilisi. Aliquam erat volutpat. Suspendisse vitae nunc nunc. Sed aliquet est suscipit sapien rhoncus non adipiscing nibh consequat. Aliquam metus urna, faucibus eu vulputate non, luctus eu justo.

Donec urna leo, vulputate vitae porta eu, vehicula blandit libero. Phasellus eget massa et leo condimentum mollis. Nullam molestie, justo at pellentesque vulputate, sapien velit ornare diam, nec gravida lacus augue non diam. Integer mattis lacus id libero ultrices sit amet mollis neque molestie. Integer ut leo eget mi volutpat congue. Vivamus sodales, turpis id venenatis placerat, tellus purus adipiscing magna, eu aliquam nibh dolor id nibh. Pellentesque habitant morbi tristique senectus et netus et malesuada fames ac turpis egestas. Sed cursus convallis quam nec vehicula. Sed vulputate neque eget odio fringilla ac sodales urna feugiat.

Phasellus nisi quam, volutpat non ullamcorper eget, congue fringilla leo. Cras et erat et nibh placerat commodo id ornare est. Nulla facilisi. Aenean pulvinar scelerisque eros eget interdum. Nunc pulvinar magna ut felis varius in hendrerit dolor accumsan. Nunc pellentesque magna quis magna bibendum non laoreet erat tincidunt. Nulla facilisi.

Duis eget massa sem, gravida interdum ipsum. Nulla nunc nisl, hendrerit sit amet commodo vel, varius id tellus. Lorem ipsum dolor sit amet, consectetur adipiscing elit. Nunc ac dolor est. Suspendisse ultrices tincidunt metus eget accumsan. Nullam facilisis, justo vitae convallis sollicitudin, eros augue malesuada metus, nec sagittis diam nibh ut sapien. Duis blandit lectus vitae lorem aliquam nec euismod nisi volutpat. Vestibulum ornare dictum tortor, at faucibus justo tempor non. Nulla facilisi. Cras non massa nunc, eget euismod purus. Nunc metus ipsum, euismod a consectetur vel, hendrerit nec nunc.	% Appendix Title
%
%%\input{Appendices/AppendixB} % Appendix Title
%
%%\input{Appendices/AppendixC} % Appendix Title
%
%\addtocontents{toc}{\vspace{2em}}  % Add a gap in the Contents, for aesthetics
%\backmatter

%% ----------------------------------------------------------------
\backmatter
\label{Bibliography}
\lhead{\emph{Bibliography}}  % Change the left side page header to "Bibliography"
\bibliographystyle{apalike}  % Use the "unsrtnat" BibTeX style for formatting the Bibliography
\bibliography{sample1}  % The references (bibliography) information are stored in the file named "Bibliography.bib"

\end{document}  % The End
%% ----------------------------------------------------------------