\lhead{\emph{Data}}  % Set the left side page header to "Abbreviations"

\chapter{Data} \label{sec:results}

\citet{issda} has made smart meter data publicly available for about 5000 households in Ireland. The Smart Metering Project was launched in 2007 with the intention of understanding consumer behaviour with regard to the influence of smart meter technology. The data used in this report are from those households which were used as controls in the trials. There are 7 weeks of data at half hourly resolution, where the weeks are labelled from 16 to 22 (inclusive). The half hours are labelled from 1 to 48 where 1 is understood to correspond to midnight. Additionally days are also numbered. The day numbering starts at 593 which is understood to be the 16th of August 2010 and the numbering ends at 641. From this, the days of the weeks, ranging from 1 to 7 where 1 is Monday and 7 is Sunday, have been deduced. Regardless of the number of occupants, each household is considered to be the unit and the terminology of ``customer'' and ``household'' should be understood as being equivalent. Being equipped with this knowledge, we explore some basic properties of the data.

\section{Profiles and Trends}
\label{subsec:basic} 
This section presents some basic visualisation and discussion of the general nature of electricity demand. First consider the histogram of the measurements shown in figure \ref{fig:hist}. The $75^{\text{th}}$ percentile of this data is 0.5 kWh so clearly most households use less than 0.5 kWh at any given time, however there are households which record almost as high as 12 kWh. These high use consumers are most likely operating a small business from home and/or may have multiple large appliances and electric vehicles. While electric vehicles may be a plausible explanation for high load in general, it is not the case here as electric vehicle recharging is a recurring, constant and prolonged activity. After a thorough search through the profiles of individual customers (not shown) such a sustained demand was not observed in the data.

\begin{figure}
\centering
\includegraphics[width=\textwidth]{usage_histogram.png}
\caption{Histogram of the half hour smart meter readings for all 503 households.}
\label{fig:hist} 
\end{figure}

Where figure \ref{fig:hist} tells us about half hourly demand, figure \ref{fig:sums} gives some general profiles. These four plots show the total/cumulative pattern of electricity demand. The top left plot in figure \ref{fig:sums} shows the dip in usage overnight, the increase for breakfast which stabilises during typical working hours and rises again for dinner with a small peak around lunch. Similarly, the top right plot of figure \ref{fig:sums} shows a recurring pattern indicating that there are specific days in the week where usage is relatively high and other days where it is relatively low. This is further confirmed by the image on the bottom left which tells us that, in total, Fridays tend to have the lowest load whereas weekends typically have the highest. The image on the bottom right shows a rise in demand starting in week 18, which is around the beginning of September, aligning with the start of the academic year for all primary and some secondary schools in Ireland. This explains why the jump in data occurs as the weeks preceding are weeks when many families may travel abroad and thus record less electric demand.

\begin{figure}
\centering
\includegraphics[width=\textwidth]{16_22_sums.pdf}
\caption{Cumulative demand profiles in kiloWatt hours (kWh) for various time horizons.}
\label{fig:sums} 
\end{figure}

It is also valuable to see how the top left profile of figure \ref{fig:sums} changes for each day of the week. From figure \ref{fig:days} it seems that there are some differences between weekdays and weekend; the breakfast peak is delayed on weekends but no categorical differences are evident for the evening peaks between weekends and weekdays. Notice that both the top left image of figure \ref{fig:sums} and the weekday profiles in figure \ref{fig:days} show three peaks: one for breakfast around 8 am, another for lunch around 1 pm and the third in the evening which is sustained for longer. While we are not currently exploring the impact and benefits of clustering, we may use these three identifiers to cluster households by their usage in the future.

\begin{figure}
\centering
\includegraphics[width=0.9\textwidth]{days_sum.png}
\caption{Total load profiles for each day of the week.}
\label{fig:days} 
\end{figure}

One way to see if there are ``extreme'' households, as a prelude to what follows in chapter \ref{sec:EVT}, is to consider the daily load of each household. This is shown in figure \ref{fig:totes}. Each marker indicates a different household though it should be noted that there is not a unique colour for each household. It is noteworthy that there is one house (coloured in light blue) that consistently appears to be using the most amount of electricity per day. It could be that this house has consistently high demand due to a small business that its occupants are operating from home.

\begin{figure}
\centering
\includegraphics[width=0.9\textwidth]{tot_daily_per_customer.png}
\caption{Total daily demand for each household.}
\label{fig:totes}
\end{figure}

One last thing to be considered in this section is the auto-correlation. It is reasonable to suspect that the demands of households are correlated to its past demand and that future weekdays will be like past weekdays and future weekends will be like past weekends. Indeed all of of our forecasts exploit this property so it is useful to verify it.  Figure \ref{fig:ty_colour} shows the relationship between the daily demand of each household, again colour coordinated, on day $d$ against the demand on day $d-1$. There seems to be evidence of a somewhat linear trend and some variation which may be resulting from the fact that weekends have not been segregated from weekdays. To see how far back this relationship holds, an autocorrelation function (fig. \ref{fig:acf_day}) is provided. The dashed line represents the 95\% confidence interval. As can be seen that there is some symmetry and while it is not shown here there is also periodicity throughout the data set though with decreasing autocorrelation. The autocorrelation function plotted in figure \ref{fig:acf_day} is an arithmetic mean of all customers, $1/n \sum_{i-1}^{n} x_i$, where $x_i$ is the load of the $i^{\text{th}}$ household, at each half hour. This gives us the empirical foundation to use the SD forecasts and others like it.

\begin{figure}
\centering
\includegraphics[width=0.9\textwidth]{autocorr_t_t-1.pdf}
\caption{Electric load day d against day d-1 in kWh.}
\label{fig:ty_colour} 
\end{figure}

\begin{figure}
\centering
\includegraphics[width=0.7\textwidth]{autocorr_day.pdf}
\caption{Auto-correlation function for 1 day. Lag is measured in half hour.}
\label{fig:acf_day} 
\end{figure}




