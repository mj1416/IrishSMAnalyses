\lhead{\emph{Extreme Value Statisitcs}}  % Set the left side page header to "Abbreviations"

\chapter{Conclusion}

We started with two objectives. The first objective was to set up a benchmark forecast and benchmark error measure. We looked at five different forecasts: the SD, LW, AA, WA and the BRR forecast.  We validated each of these forecasts using 2 error measures, the $4^{\text{th}}-$ norm error and the adjusted $4^{\text{th}}-$ norm error. In this report, we had started with the SD forecast which we found to be, on average,  the best of the five forecasts or the second best and due to its popularity in the literature, this was thought to be good candidate for the benchmark forecast. Another good candidate is the AA forecast since at least qualitatively, but also quantitatively if the adjusted $4^{\text{th}}-$ norm error is used, the best forecast; it produces a forecast where the peaks are better represented than the others. Forecasts such as the WA and the BRR suffered due to the over-smoothing of the profiles and the LW forecast does not allow for week-to-week variability. Due to the widespread usage and transparency, both the SD forecast and the $4^{\text{th}}-$ norm error are established as the benchmark. It should thus be noted that using other forecasts (AA) and error measures (adjusted $4^{\text{th}}-$ norm), we established better forecasts and better methods of judging the goodness of said forecasts during this project.

The second objective was to apply results from extreme value theory to electric load data. We used both the BM method and the POT method to estimate the EVI and the BM method to estimate the right endpoint for weekly maxima. From this, we gained that weekly maxima are light tailed and have a right endpoint between 12 and 13.5 kWh. If longer historic data is available, DNOs can do the same analysis for each household to estimate the right endpoint thereby generating personalised electricity contracts and limits. Similarly, business can evaluate if they have enough headroom in their current plans and whether they should invest in electricity storing devices. We then went further and looked the frequency of extremes by relaxing the assumption of identically distributed. From this we established the presence of heteroscedastic extremes, meaning that half hourly maxima are not identically distributed. Crucially we looked at the scedasis function. Again DNOs can look at the scedasis function for each household and identify any major changes to usage such as purchase of electric vehicles, PV cells, etc which is significant when it comes to demand response and customer service. Finally we started to connect forecasting with EVT by considering the scedasis function of forecasting errors. From this we learnt that high usage was inherently linked to high forecasting errors and subsequently concluded that perhaps introducing clustering may be one way to tackle this problem.

While we have set a benchmark and made improvements on it, it has come at a price. Both the AA forecast and the adjusted $4^{\text{th}}-$ norm error measure in their current implementation are time consuming. Given the strength of the forecast and the error measure, it is viable and beneficial to find an alternative (to the Hungarian algorithm) to solve the minimisation problem for the local permutations in time. Not only, this we have not yet introduced impacts of temperature, rainfall or modelled social and demographic factors. This is a valuable venture as we want the forecast to be realistic and not just reflect the typical day. By combining existing forecasts with EVT may help us to generate forecasts that represent the peaks in load accurately. Knowing when peaks are going to occur and the magnitude of those peaks will help DNOs be prepared for them. Without knowing when peaks in demand will occur and how large those peaks will be means that DNOs will always have to ensure that the network can supply as much during non-peak hours of the day as during the peak hours of the day. However with analyses such as ours, DNOs can do better demand response and thus reduce both consumption and subsequently generation thereby reducing overall carbon emissions. %tpeak load is can use this information to effectively incentivise customers to spread their usage over longer periods of the day. It is important to realise that DNOs must be able to provide the maximum demand at any given time to avoid the possibility of a power outage which means that even during non-peak

While the application of EVT to electric load data is novel, everything we have presented so far is non-parametric. Since we are motivated by industrial applications and want to reflect reality as closely as possible, it is important and necessary to introduce the parametric framework in the adaptation of EVT to electric load. This is the natural progress for this work. Combining this with the modelling we spoke of before, it will be possible to do climate change experiments and see how the EVI, right endpoint, etc. may change in the future. Results such as this will help DNO and policy makers to conduct timely and cost effective upgrade and maintenance to the electric. Ultimately we want our analyses to help make informed decisions and we want to have confidence in these analyses especially when we are uncertain about the future. It is also important that the decisions support local and national policy to combat anthropogenic climate change by collectively reducing carbon emissions.